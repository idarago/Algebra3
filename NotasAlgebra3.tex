\documentclass[12pt]{book}

% This first part of the file is called the PREAMBLE. It includes
% customizations and command definitions. The preamble is everything
% between \documentclass and \begin{document}.

\usepackage[margin=1in]{geometry}  % set the margins to 1in on all sides
\usepackage{graphicx}              % to include figures
\usepackage{amsmath}               % great math stuff
\usepackage{amsfonts}              % for blackboard bold, etc
\usepackage{amsthm}                % better theorem environments
\usepackage{amssymb}
\usepackage[activeacute, spanish]{babel}
\usepackage[utf8]{inputenc}
\usepackage{hyperref}
\usepackage{multicol}
\usepackage{mathrsfs}
\usepackage{tikz-cd}
\usetikzlibrary{calc}
\usetikzlibrary{matrix}

\setcounter{tocdepth}{3}% to get subsubsections in toc

\let\oldtocsection=\tocsection

\let\oldtocsubsection=\tocsubsection

\let\oldtocsubsubsection=\tocsubsubsection

% various theorems, numbered by section

\newtheorem{teo}{Teorema}[section]
\newtheorem{lem}[teo]{Lema}
\newtheorem{prop}[teo]{Proposición}
\newtheorem{cor}[teo]{Corolario}

\theoremstyle{definition}
\newtheorem{conj}[teo]{Conjetura}
\newtheorem{obs}[teo]{Observación}
\newtheorem{defn}[teo]{Definición}
\newtheorem{ax}[teo]{Axioma}
\newtheorem{ex}[teo]{Ejemplo}

\newcommand{\bd}[1]{\mathbf{#1}}  % for bolding symbols
\newcommand{\CC}{\mathbb{C}}
\newcommand{\RR}{\mathbb{R}}      % for Real numbers
\newcommand{\ZZ}{\mathbb{Z}}      % for Integers
\newcommand{\NN}{\mathbb{N}}
\newcommand{\QQ}{\mathbb{Q}}
\newcommand{\FF}{\mathbb{F}}
\newcommand{\col}[1]{\left[\begin{matrix} #1 \end{matrix} \right]}
\newcommand{\comb}[2]{\binom{#1^2 + #2^2}{#1+#2}}
\newcommand{\eps}{\varepsilon}
\renewcommand{\hom}{\mathrm{Hom}}
\let\oldemptyset\emptyset
\let\emptyset\varnothing
\DeclareMathOperator{\id}{id}
\DeclareMathOperator{\mcm}{mcm}
\DeclareMathOperator{\mcd}{mcd}
\DeclareMathOperator{\ord}{ord}
\DeclareMathOperator{\im}{im}
\DeclareMathOperator{\End}{End}
\DeclareMathOperator{\Aut}{Aut}
\DeclareMathOperator{\sg}{sg}
\DeclareMathOperator{\coker}{coker}
\DeclareMathOperator{\Obj}{Obj}
\DeclareMathOperator{\rank}{rk}
\DeclareMathOperator{\gr}{gr}
\DeclareMathOperator{\car}{car}
\DeclareMathOperator{\Nil}{Nil}
\DeclareMathOperator{\spec}{Spec}
\DeclareMathOperator{\ev}{ev}
\DeclareMathOperator{\ann}{Ann}
\DeclareMathOperator{\Gal}{Gal}
\def\acts{\curvearrowright}
\def\stca{\curvearrowleft}



\begin{document}
\pagenumbering{roman}\clearpage\tableofcontents\newpage\pagenumbering{arabic}

\chapter{Teoría de Cuerpos}

\section{Definiciones Básicas}

Nuestro interés va a radicar en la Teoría de Cuerpos. Una primera pregunta natural para preguntarse es, si dado un anillo conmutativo $A$ existe un cuerpo $k$ y un morfismo inyectivo de anillos $\iota:A\hookrightarrow k$. Si existieran $a,b\in A-\{0\}$ tales que $ab=0$ entonces $0=\iota(0)=\iota(ab)=\iota(a)\iota(b)$, con $\iota(a)\neq 0$ y $\iota(b)\neq 0$ por la inyectividad de $\iota$, y esto es absurdo por ser $k$ un cuerpo. Entonces es necesario que $A$ sea un dominio íntegro. Pero veamos que esto es suficiente. En efecto, definimos $k=\mathrm{Frac}(A)$ su cuerpo de fracciones con el morfismo canónico $\iota:A\to k$ dado por $\iota(a) = \dfrac{a}{1}$. Además, la propiedad universal del cuerpo de fracciones nos dice que $\mathrm{Frac}(A)$ es el cuerpo más chico que cumple esta propiedad. Es decir, si $f:A\to B$ es un morfismo de anillos tal que $f(A-\{0\})\subseteq \mathcal{U}(B)$, entonces existe un único $\overline{f}:\mathrm{Frac}(A)\to B$ tal que el siguiente diagrama conmuta:
\begin{center}
\begin{tikzcd}[row sep=3.3em,column sep=4em,minimum width=2em]
 A\arrow[]{d}[left,font=\normalsize]{\iota}\arrow[]{r}[above, font=\normalsize]{f}& B \\
\mathrm{Frac}(A)\arrow[dashed]{ur}[right, font=\normalsize]{\overline{f}} & \\
\end{tikzcd}
\end{center}

Además, sabemos que $\ZZ$ es un objeto inicial en la categoría de anillos. Es decir, para cualquier anillo $A$, existe un único morfismo de anillos $\lambda:\mathbb{Z}\to A$, y podemos mirar así el núcleo de este morfismo. Si $\ker \lambda = 0$ entonces por el Primer Teorema de Isomorfismo debemos tener que $\ZZ \simeq \im \lambda\subseteq A$. Por lo tanto, si $A=k$ es un cuerpo, usando la propiedad universal del cuerpo de fracciones obtenemos un morfismo $f:\mathbb{Q}\to k$ que debe ser inyectivo pues $\mathbb{Q}$ es un cuerpo (y así sus ideales son sólo los triviales). Entonces, si $\ker\lambda = 0$ tenemos que $\mathbb{Q}$ es el mínimo subcuerpo en el sentido de inclusión en $k$. En este caso se dice que $k$ tiene característica $0$.

En el caso que $\ker\lambda \neq 0$, como debe ser un ideal de $\ZZ$, debemos tener $\ker \lambda = n\ZZ$ con $n>1$. Entonces, nuevamente por el Teorema de Isomorfismo se tiene que $\ZZ/\ker\lambda \simeq \im\lambda\subseteq k$. Esto quiere decir que $\ZZ/n\ZZ$ es un dominio íntegro pues está metido en un cuerpo. Pero esto a su vez implica que el ideal $n\ZZ$ debe ser primo, y así como $\ZZ$ es un Dominio de Ideales Principales, el ideal $n\ZZ$ debe ser maximal. Entonces $\ZZ_p$ es un cuerpo que está metido en $k$ y es el más chico en el sentido de inclusión. En este caso se dice que $k$ es un cuerpo de característica $p$.

Vamos a notar $\ZZ_p = \FF_p$. Dado un cuerpo $k$ entonces se define su \textbf{cuerpo primo} como el mínimo subcuerpo contenido en $k$ en el sentido de la inclusión. Si $k$ es de característica $0$, su cuerpo primo es $\QQ$ y si es de característica $p$ su cuerpo primo es $\FF_p$.

Sea $A\subseteq B$ un subanillo y $b\in B$. Entonces $A[b]$ es la mínima $A$-subálgebra de $B$ que contiene a $b$. Si definimos el morfismo evaluación $\mathrm{ev}_b:A[x]\to B$ dado por $\mathrm{ev}_b(f) = f(b)$, entonces $A[b] = \im \mathrm{ev}_b$. Por lo tanto, por el Primer Teorema de Isomorfismo, tenemos que $A[b]\simeq A[x]/\ker \mathrm{ev}_b$.

\begin{defn}
Una \textbf{extensión de cuerpos} $F/k$ consiste de un cuerpo $k$ y un cuerpo $F$ tal que $k$ es un subcuerpo de $F$.
\end{defn}

\begin{obs}
Sea $F/k$ una extensión. Entonces $F$ es una $k$-álgebra. En particular, $F$ es un $k$-espacio vectorial. Se llama el \textbf{grado de la extensión} a la dimensión de $F$ como $k$-espacio vectorial y se nota $[F:k] = \dim_k F$.
\end{obs}

\begin{defn}
Una extensión $F/k$ se dice \textbf{finita} si $[F:k]<+\infty$. Si no, se dice infinita.
\end{defn}

\begin{obs}
Si $[F:k]=1$ entonces $F=k$. Esto es obvio pues los $k$-espacios vectoriales tienen noción de rango. Si $\CC$ es el cuerpo de los complejos, entonces $[\CC:\RR]=2$, pues en efecto, $\{1,i\}$ es una base de $\CC$ como $\RR$-espacio vectorial. Sin embargo, tanto $\RR/\QQ$ como $\CC/\QQ$ son extensiones infinitas pues $\QQ$ es numerable y así cualquier $\QQ^n$ con $n\in\mathbb{N}$, mientras que tanto $\RR$ como $\CC$ no lo son.
\end{obs}

\begin{defn}
Sea $F/k$ una exntesión de cuerpos y $\alpha\in F$. Decimos que $\alpha$ es \textbf{algebraico} sobre $k$ si existe $f\in k[x]$ no nulo tal que $f(\alpha)=0$. Equivalentemente, $\mathrm{ev}_\alpha:k[x]\to F$ no es inyectivo (ie. $\ker \ev_\alpha \neq 0$).
\end{defn}

\begin{obs}
Sea $\alpha\in F$ algebraico sobre $k$. Entonces $\ker\ev_\alpha\subseteq k[x]$ es un ideal no nulo. Como $k[x]$ es un DIP, existe $p\in k[x]$ tal que $\langle p\rangle = \ker\ev_\alpha$. Por otra parte, como $k[x]/\langle p\rangle \simeq k[\alpha]\subseteq F$ y $k[\alpha]$ es un subanillo de $F$ y $F$ es cuerpo, $k[\alpha]$ debe ser dominio íntegro y así $\langle p\rangle$ debe ser un ideal primo. Pero como $k[x]$ es DIP, esto implica que $p$ es un polinomio irreducible, y además $p\mid g$ para todo $g\in k[x]$ tal que $g(\alpha)=0$. Por otra parte, puedo tomar a $p$ como mónico pues $k$ es cuerpo (simplemente divido por el coeficiente principal y eso no altera al ideal). Si $g\in k[x]$, $g$ es irreducible y $g(\alpha)=0$, entonces $p\mid g$, pero como $g$ es irreducible, debemos tener $g=\lambda p$ con $\lambda \in k^\times$.

Entonces, si $\alpha$ es algebraico sobre $k$, existe un único $p\in k[x]$ irreducible y mónico tal que $p(\alpha)=0$. A este polinomio $p$ se lo llama el \textbf{polinomio minimal} de $\alpha$ en $k$ y se lo denota por $f(\alpha,k), m(\alpha,k)$ o $\mathrm{Irr}(\alpha,k)$.

Entonces $m(\alpha,k)$ cumple que es el único irreducible y mónico que anula a $\alpha$, es el único mónico que cumple que $p\mid g$ para todo $g\in k[x]$ tal que $g(\alpha)=0$ y además es el polinomio mónico de grado mínimo que anula a $\alpha$.
\end{obs}

\begin{obs}
Sea $F/k$ una extensión y $\alpha\in F$ algebraico sobre $k$. Entonces $k[x]/\ker \ev_\alpha \simeq k[\alpha]\subseteq F$. Como $F$ es cuerpo, $k[\alpha]$ debe ser dominio íntegro y así $\ker\ev_\alpha$ es un ideal primo, que es no nulo por ser $\alpha$ algebraico. Como $k[x]$ es un DIP, debemos tener entonces que $\ker\ev_\alpha$ es un ideal maximal y así $k[\alpha]$ es un cuerpo. Esto nos permite concluir que por ejemplo $\QQ[\sqrt{2}]$ y $\QQ[\sqrt[3]{5}]$ son cuerpos.
\end{obs}

\begin{defn}
Sea $F/k$ una extensión y $\alpha\in F$. Definimos $k(\alpha)$ como el mínimo subcuerpo de $F$ que contiene a $k$ y $\alpha$.
\end{defn}

\begin{obs}
Notemos que $k\subseteq k[\alpha]\subseteq k(\alpha)\subseteq F$. Es decir, el mínimo subcuerpo que contiene a $k$ y $\alpha$ contiene al mínimo subanillo que contiene a $k$ y $\alpha$. Entonces, como $k[\alpha]\subseteq F$ debe ser dominio íntegro, y así $k(\alpha)$ debe ser su cuerpo de fracciones. Esto implica que $k(\alpha)=\left\{\dfrac{f(\alpha)}{g(\alpha)} : f,g\in k[x], g(\alpha)\neq 0 \right\}$.
\end{obs}

\begin{prop}
Sea $F/k$ una extensión y $\alpha\in F$. Entonces $k[\alpha] = k(\alpha)$ si y sólo si $\alpha$ es algebraico sobre $k$. En ese caso, $[k(\alpha):k] = \gr(m(\alpha,k))$.
\begin{proof}
$(\Longleftarrow)$ Si $\alpha$ es algebraico, ya vimos que $k[\alpha]$ es un cuerpo y así $k[\alpha]=k(\alpha)$.
$(\Longrightarrow)$ Si $k[\alpha]=k(\alpha)$ y $\alpha$ no es algebraico, entonces $\ker\ev_\alpha =0$ y así $k[\alpha]\simeq k[x]$. Pero $k[x]$ no es un cuerpo mientras que $k[\alpha]=k(\alpha)$ sí lo es. Absurdo.

Para probar la segunda afirmación, sea $p=m(\alpha,k)$ el polinomio minimal y $d=\gr p$ su grado. Veamos que $\mathcal{B}=\{1,\alpha,\alpha^2,\ldots,  \alpha^{d-1}\}$ es una base de $k(\alpha)$ como $k$-espacio vectorial. Primero, veamos que $\mathcal{B}$ es linealmente independiente. En efecto, sean $a_i\in k$ tales que $a_0 + a_1 \alpha + \ldots + a_{d-1}\alpha^{d-1}=0$. Sea $g(x) = a_0 + a_1x + \ldots + a_{d-1}x^{d-1}$. Como $g(\alpha)=0$ y $\gr g < \gr p$, debemos tener que $g=0$ y así $a_i =0$ para todo $i$. Esto implica que $\mathcal{B}$ es linealmente independiente. Resta ver que $\mathcal{B}$ genera. Como $k(\alpha)=k[\alpha]$, si $x\in k(\alpha)$ entonces $x=f(\alpha)$ para algún $f\in k[x]$. Por algoritmo de división, $f=ph+r$ con $r=0$ o $\gr r < d$. Entonces, $f(\alpha) = p(\alpha)h(\alpha) + r(\alpha) = r(\alpha)$. Entonces $x=f(\alpha)=r(\alpha) = b_0 + b_1\alpha + \ldots + b_{d-1}\alpha^{d-1}$, y estamos.
\end{proof}
\end{prop}

\begin{defn}
Sea $F/k$ una extensión de cuerpos. $\alpha\in F$ se dice \textbf{trascendente} sobre $k$ si no es algebraico.
\end{defn}

\begin{obs}
Notar que $\alpha$ es trascendente sobre $k$ si y sólo si $k[\alpha]\simeq k[x]$.
\end{obs}

\begin{defn}
Una extensión $F/k$ se dice \textbf{algebraica} si todo $\alpha\in F$ es algebraico sobre $k$.
\end{defn}

\begin{prop}
Si $F/k$ es una extensión finita entonces es algebraica.
\begin{proof}
Sea $\alpha\in F$. Debo ver que es algebraico sobre $k$. Como $[F:k]<+\infty$ existe un $n\in \NN$ tal que $\{1,\alpha,\ldots , \alpha^{n}\}$ resulta linealmente dependiente, y así existen $a_i\in k$ no todos cero tales que $a_0 + a_1 \alpha + \ldots + a_{n}\alpha^{n} = 0$ y así tengo un polinomio no nulo que anula a $\alpha$.
\end{proof}
\end{prop}

\begin{obs}
La recíproca es falsa: es decir, que una extensión $F/k$ sea algebraica no implica $[F:k]<+\infty$. Un ejemplo (que más adelante probaremos) es $\overline{\QQ}/\QQ$ donde se define $\overline{\QQ} = \{a\in\CC : a \text{ es algebraico sobre }\QQ\}$.
\end{obs}

\begin{defn}
Una \textbf{torre de extensiones} es una cadena de contenciones de subcuerpos $k_1\subseteq \ldots \subseteq k_n$.
\end{defn}

\begin{prop}
Si $k\subseteq F\subseteq E$ es una torre de extensiones, entonces $[E:k] = [E:F][F:k]$.
\begin{proof}
Sea $\{x_i\}_{i\in I}$ una base de $E$ como $F$-espacio vectorial. Entonces, $[E:F] = \sharp I$. Sea $\{y_j\}_{j\in J}$ una base de $F$ como $k$-espacio vectorial. Entonces, $[F:k]=\sharp J$. Afirmo que $\{x_i y_j\}_{i\in I, j\in J}$ es una base de $E$ como $k$-espacio vectorial. Veamos que genera primero: Si $x\in E$ entonces $x=\displaystyle\sum_{i} \alpha_i x_i = \displaystyle\sum_{i}\sum_{j} \alpha_{ij}y_j x_i$, donde la segunda igualdad se da porque $\alpha_i \in F$. Ahora, veamos que son linealmente independientes. Si $\displaystyle\sum_{i,j}c_{ij}x_iy_j=0$ entonces debemos tener que $\displaystyle\sum_{i}\left(\sum_{j}c_{ij}y_j\right) x_i = 0$ y así como $\{x_i\}_{i\in I}$ es base, cada $\displaystyle\sum_{j}c_{ij}y_j=0$ y así como $\{y_j\}_{j\in J}$ es base debemos tener que $c_{ij}=0$ para todo $i,j$. Y listo.
\end{proof}
\end{prop}

\begin{defn}
Sea $F/k$ una extensión de cuerpos y $S\subseteq F$ un subconjunto. Entonces $k[S]$ es la mínima $k$-subálgebra de $F$ que contiene a $S$. Además, $k(S)$ es el mínimo subcuerpo de $F$ que contiene a $k$ y $S$. Entonces $k\subseteq k[S]\subseteq k(S)\subseteq F$. Como $k[S]\subseteq F$ es un dominio íntegro, debemos tener que $k(S)$ es su cuerpo de fracciones. Si $S=\{\alpha_1,\ldots , \alpha_m\}$ entonces $k[S] = k[\alpha_1,\ldots , \alpha_m] = \{f(\alpha_1,\ldots , \alpha_m) : f\in k[x_1,\ldots , x_m]\}$. Por lo tanto, su cuerpo de fracciones es $k(S) = k(\alpha_1,\ldots , \alpha_m) = \left\{ \dfrac{f(\alpha_1,\ldots , \alpha_m)}{g(\alpha_1,\ldots ,\alpha_m)} : f,g\in k[x_1,\ldots , x_m], g(\alpha_1,\ldots , \alpha_m)\neq 0 \right\}$.
\end{defn}

\begin{obs}
Sea $k(\alpha_1,\alpha_2)$. Entonces $k(\alpha_1,\alpha_2) = (k(\alpha_1))(\alpha_2) = (k(\alpha_2))(\alpha_1)$. Es decir, puedo ir agregando las raíces de a una.
\end{obs}

\begin{obs}
Sea $k\subseteq F\subseteq E$ una torre de extensiones y $\alpha\in E$ algebraico sobre $k$. Entonces $\alpha$ es algebraico sobre $F$ y $m(\alpha,F)\mid m(\alpha,k)$ como polinomios en $F[x]$. Un ejemplo es $x^2 - 5$ es el minimal de $\sqrt{5}$ en $\QQ$ mientras que en $\RR$ tenemos que $x-\sqrt{5}$ es su minimal.
\end{obs}

\begin{defn}
Sea $A$ un dominio íntegro. Decimos que un polinomio $p\in A[x]$ es \textbf{primitivo} si los únicos elementos de $A$ que dividen a todos los coeficientes de $p$ son las unidades.
\end{defn}

\begin{obs}
Sea $A$ un DFU y $k$ su cuerpo de fracciones. Si $f(x)\neq 0\in k[x]$ entonces $f(x) = \gamma f_1(x)$ con $\gamma\in k$ y $f_1(x)$ un polinomio primitivo sobre $A[x]$. Más aún, esta factorización es única a menos de unidades en $A$. En efecto, si $f(x) = \alpha_0 + \alpha_1 x + \ldots + \alpha_n x^n$ donde $\alpha_i\in k$ y $\alpha_n\neq 0$. Podemos escribir $\alpha_i = a_i b_i^{-1}$ para ciertos $a_i,b_i\in A$. Por lo tanto, si $b = b_1\cdots b_n$, tenemos que $bf(x) = cf_1(x)$ donde $f_1(x)\in A[x]$ es un polinomio primitivo. Esto quiere decir que $f(x) = \gamma f_1(x)$ donde $\gamma = cb^{-1}\in k$. Ahora, si $f(x)=\overline{\gamma}f_2(x)$, entonces $\overline{\gamma} = de^{-1}$, y así $cef_1(x) = bdf_2(x)$. Pero esto quiere decir que $f_1$ y $f_2$ son asociados y $\overline{\gamma} = u\gamma$ con $u$ unidad, como queríamos.

De esto se deduce inmediatamente que si $f,g\in A[x]$ son primitivos, y son asociados en $k[x]$, entonces deben ser asociados en $A[x]$. En efecto, como $f(x) = \alpha g(x)$ con $k\ni\alpha\neq 0$ y por la unicidad de la escritura, $\alpha$ debe ser unidad en $A$.
\end{obs}

\begin{obs}
El producto de polinomios primitivos es primitivo en un $A$ DFU. Sean $f(x),g(x)$ polinomios primitivos, pero $h(x)=f(x)g(x)$ no lo es. Entonces, existe algún elemento irreducible (equiv. primo por ser $A$ DFU) $p\in A$ tal que $p\nmid f(x), p\nmid g(x)$ pero $p\mid h(x)$. Como $p$ es primo, esto implica que $A/\langle p\rangle$ es un dominio íntegro, y así $A/\langle p\rangle [x]$ es un dominio íntegro también. Si aplicamos el morfismo canónico de proyección $A[x]\to A/\langle p\rangle [x]$ (es decir, el único morfismo que está definido por $a\in A\mapsto \overline{a}\in A/\langle p\rangle$ y $x\mapsto x$ en virtud de la propiedad universal del anillo de polinomios) tenemos que $\overline{f}(x)\overline{g}(x) = \overline{h}(x)=\overline{0}$. Pero $\overline{f}(x)\neq \overline{0}$ y $\overline{g}(x)\neq \overline{0}$, contradiciendo que $A/\langle p\rangle$ sea un dominio íntegro.
\end{obs}

\begin{lem}[Lema de Gauss]
Sea $A$ un DFU y $k$ su cuerpo de fracciones. Si $f$ es irreducible en $A[x]$ entonces $f$ es irreducible en $k[x]$.
\begin{proof}
Supongamos que $f(x)\in A[x]$ tiene grado positivo y es irreducible en $A[x]$. Esto quiere decir que $f$ es primitivo (pues si no separamos algún divisor de todos los coeficientes). Supongamos que $f$ es reducible en $k[x]$ y escribamos $f(x) = \varphi_1(x)\varphi_2(x)$ con $\varphi_1,\varphi_2\in k[x]$. Podemos escribir entonces $\varphi_1(x) = \alpha_1 f_1(x)$ con $\alpha_1\in k$ y $f_1$ primitivo en $A[x]$ (y lo mismo para $\varphi_2(x)=\alpha_2 f_2(x)$). Entonces, $f(x) = \alpha_1\alpha_2 f_1(x)f_2(x)$. Como el producto de polinomios primitivos es primitivo. Por la observación anterior, debemos tener que $f(x)$ y $f_1(x)f_2(x)$ son asociados en $A[x]$ pues lo son en $k[x]$. Esto contradice la irreducibilidad de $f(x)$ en $A[x]$. El lema sigue.
\end{proof}
\end{lem}

\begin{teo}
Si $A$ es un DFU entonces $A[x]$ también es un DFU.
\begin{proof}

Sea $f(x)\in A[x]$ un polinomio no nulo y que no es unidad. Sea $k$ el cuerpo de fracciones de $A$. Entonces, podemos escribir $f(x) = d f_1(x)$ con $d\in k$ y $f_1$ primitivo. Si $\gr f_1(x) >0$ y $f_1(x)$ no es una unidad ni es irreducible, tendremos que $f_1(x) = f_{11}(x)f_{12}(x)$ con $\gr f_{11},\gr f_{12} > 0$. Claramente los dos polinomios deben ser primitivos, pues si alguno no lo fuera (sin pérdida de la generalidad $f_{11}$) tendríamos $a\mid f_{11}$ entonces $a\mid f$ (pues divide a todos los coeficientes), lo que implica $a$ unidad y así $f_{11}$ primitivo. Por lo tanto, si procedemos por inducción sobre el grado podemos escribir $f_1(x) = q_1(x)\cdots q_t(x)$ con $q_i(x)\in A[x]$ irreducibles. Si $d$ no es una unidad, escribimos $d=p_1\cdots p_s$ con $p_i\in A$ irreducibles. Claramente, estos son irreducibles en $A[x]$. Utilizando las factorizaciones de $d$ y $f_1(x)$, obtenemos una factorización de $f(x)$. 

Sólo resta probar la unicidad de la factorización a menos de unidades y reordenamiento. Supongamos primero que $f(x)$ es primitivo. Entonces, sus factores irreducibles tienen grado positivo. Tenemos $f(x) = q_1(x)\cdots q_h(x) = q_1'(x)\cdots  q_k'(x)$ donde los $q_i,q_j'$ son irreducibles de grado positivo. Entonces, estos son irreducibles en $k[x]$ por el Lema de Gauss. Como $k[x]$ es un DFU (pues es un dominio euclídeo al ser $k$ cuerpo), debemos tener que $h=k$ y que existe una permutación $\sigma\in S_k$ tal que $q_i'(x)$ y  $q_{\sigma(i)}(x)$ son asociados en $k[x]$ (y por lo tanto en $A[x]$ como vimos en una observación anterior). Por lo tanto, la unicidad en el caso que $f(x)$ es primitivo sigue. Supongamos ahora que $f(x)$ no es primitivo. Entonces, toda factorización en irreducibles en $A[x]$ contiene factores en $A$. Como vimos en una observación anterior, esa parte en $A$ es única salvo unidades. Como $A$ es un DFU, esa parte en $D$ tiene factorización única. Ahora, la parte que resta (si es que resta algo) es un polinomio primitivo. Pero para los polinomios primitivos ya probamos que la factorización es única. Y estamos. 

\end{proof}
\end{teo}

\begin{cor}
Si $A$ es un DFU entonces $A[x_1,\ldots , x_n]$ también lo es.
\begin{proof}
Simplemente es proceder por inducción sobre la cantidad de variables.
\end{proof}
\end{cor}

\begin{obs}
Sabemos que DIP implica DFU. Veamos que la recíproca es falsa. En efecto, como $\ZZ$ es DFU, su anillo de polinomios $\ZZ[x]$ también lo es. Pero $\ZZ[x]$ no es DIP pues el ideal $\langle 2,x\rangle$ no es principal.
\end{obs}

\begin{lem}[Criterio de Eisenstein]
Sea $A$ un DFU y $k$ su cuerpo de fracciones. Si $f\in A[x]$, $f=a_0 + a_1x + \ldots + a_n x^n$ y se cumple que para algún primo $p\in A$, $p\nmid a_n$, $p\mid a_i$ para todo $0\leq i <n$ y $p^2\nmid a_0$, entonces $f$ resulta irreducible en $k[x]$.
\begin{proof}

Supongamos que $f(x)$ fuera reducible sobre $A[x]$. Supongamos que existen polinomios $g(x),h(x)\in A[x]$ de grado positivo de modo tal que $f(x)=g(x)h(x)$. Luego, si $g(x) = b_0 + b_1 x + \ldots + b_m x^m$ ($b_m\neq 0$) y $h(x)=c_0 + c_1x+\ldots + c_k x^k$ ($c_k\neq 0$), tenemos que $a_0 = b_0 c_0$. Como $p\mid a_0$ y es primo, debemos tener que $p\mid b_0$ o $p\mid c_0$. Pero no pueden pasar ambas alternativas a la vez, pues si no $p^2 \mid b_0c_0=a_0$. Supongamos sin pérdida de generalidad que $p\mid b_0$ y $p\nmid c_0$. Además, notemos que $a_n = b_m c_k$ y como $p\nmid a_n$, debemos tener $p\nmid b_m$. Sea $r$ el menor índice tal que $p\nmid b_r$. Entonces, $a_r = (b_0c_r + \ldots + b_{r-1}c_1) + b_rc_0$, con $r\leq m < m+k=n$. Entonces, $p\mid a_r$. Pero $p$ divide a todo el paréntesis y $p\nmid b_r c_0$ pues no divide a ninguno de los dos. Absurdo, que provino de suponer que $f$ fuera reducible.

Si $f$ no se puede escribir como producto de polinomios de grado positivo, tenemos que $f(x) = \alpha g(x)$ con $\alpha\in A$ y $g$ primitivo. Si $g$ es primitivo, entonces haciendo el mismo argumento que antes, debe ser irreducible en $A[x]$ y por Lema de Gauss, en $k[x]$. Entonces, $f(x)$ es asociado a un irreducible en $k[x]$ y así irreducible en $k[x]$. Y estamos.

\end{proof}
\end{lem}

\begin{ex}
Supongamos que queremos calcular $[\QQ(i,\sqrt{2}),\QQ]$. Notemos que $x^2 -2$ es el minimal de $\sqrt{2}$ en $\QQ$, ya que es irreducible por Criterio de Eisenstein y anula a $\sqrt{2}$. Entonces, $[\QQ(\sqrt{2}),\QQ]=2$. Ahora, notemos que $m(i,\QQ(\sqrt{2})) \mid m(i,\QQ) = x^2 + 1$. Entonces, $\gr m(i,\QQ(\sqrt{2})) \leq 2$. Pero como $i\notin \QQ(\sqrt{2})$, debe tener grado $2$. Esto quiere decir que $[\QQ(\sqrt{2})(i):\QQ(\sqrt{2})] = 2$. Por lo tanto, $[\QQ(i,\sqrt{2}):\QQ]=4$.

Ahora, supongamos que queremos calcular $[\QQ(\sqrt[4]{5},\sqrt[5]{3}) : \QQ]$. Vemos que $[\QQ(\sqrt[4]{5}):\QQ]=4$ pues el minimal de $\sqrt[4]{5}$ sobre $\QQ$ es $x^4 - 5$ (pues nuevamente, lo anula y es irreducible por Criterio de Eisenstein). De forma similar, $[\QQ(\sqrt[5]{3}):\QQ]=5$ pues el minimal de $\sqrt[5]{3}$ es $x^5 - 3$ (lo anula y es irreducible por Criterio de Eisenstein nuevamente). Entonces, sabemos que $4$ y $5$ dividen al grado de la extensión original, y como son coprimos, $20$ debe hacerlo. Pero como el polinomio minimal $m(\sqrt[4]{5},\QQ(\sqrt[5]{3}) \mid m(\sqrt[4]{5},\QQ)$, tenemos que el grado de la extensión está acotado $[\QQ(\sqrt[5]{3})(\sqrt[4]{5}):\QQ(\sqrt[5]{3})]\leq [\QQ(\sqrt[4]{5}):\QQ]=4$. Entonces, esto nos dice que $[\QQ(\sqrt[4]{5},\sqrt[5]{3}) : \QQ] = [\QQ(\sqrt[5]{3})(\sqrt[4]{5}):\QQ(\sqrt[5]{3})] [\QQ(\sqrt[5]{3}):\QQ] \leq 20$. Pero como $20\mid [\QQ(\sqrt[4]{5},\sqrt[5]{3}) : \QQ]$, debe ser $20\leq [\QQ(\sqrt[4]{5},\sqrt[5]{3}) : \QQ]$. Por lo tanto, tenemos ambas desigualdades y así debe valer la igualdad.

Finalmente, supongamos que queremos calcular el polinomio minimal de $\sqrt{2}+\sqrt[3]{3}$ sobre $\QQ$. Escribimos $\alpha = \sqrt{2}+\sqrt[3]{3}$. Entonces, $\alpha-\sqrt{2} = \sqrt[3]{3}$. Elevando al cubo a ambos lados tenemos que $(\alpha-\sqrt{2})^3 = 3$. Es decir, $\alpha^3 - 3\alpha^2 \sqrt{2} + 6\alpha - 2\sqrt{2} = 3$. Entonces $\alpha^3 + 6\alpha - 3 = \sqrt{2}(3\alpha^2 + 2)$. Esto en particular nos dice que $\sqrt{2}\in\QQ(\alpha)$, y así $\sqrt[3]{3}\in\QQ(\alpha)$. Elevando al cuadrado esa identidad y pasando restando los términos, se tiene que un polinomio de grado $6$ anula a $\alpha$. Si vemos que es irreducible ya estaremos. Pero para eso, a su vez, nos alcanzará probar que $[\QQ(\alpha):\QQ]=6$. Notemos que $\QQ(\sqrt{2}+\sqrt[3]{3}) = \QQ(\sqrt{2},\sqrt[3]{3})$. En efecto, la contención $\subseteq)$ es obvia y la otra contención la probamos unos renglones más arriba. Además, como $[\QQ(\sqrt{2}):\QQ]= 2$ y $[\QQ(\sqrt[3]{3}):\QQ]=3$, y son coprimos, $6\mid [\QQ(\sqrt{2},\sqrt[3]{3}):\QQ]$. esto implica que $[\QQ(\sqrt{2},\sqrt[3]{3}):\QQ]\geq 6$. Pero como tenemos un polinomio de grado $6$ que anula a $\alpha$, debe ser el minimal, y así irreducible.

\end{ex}

\begin{defn}
Una extensión $F/k$ se dice \textbf{finitamente generada} si $F=k(\alpha_1,\ldots , \alpha_m)$ para ciertos $\alpha_i\in F$.
\end{defn}
\begin{defn}
Una extensión $F/k$ se dice \textbf{simple} si $F=k(\alpha)$ para algún $\alpha\in F$.
\end{defn}

\begin{obs}
Notemos que $\QQ(\pi)/\QQ$ es una extensión simple (pero no es algebraica). Además, $\QQ(\sqrt{2},\sqrt[3]{3})/\QQ$ es simple pues como vimos $\QQ(\sqrt{2},\sqrt[3]{3})=\QQ(\sqrt{2}+\sqrt[3]{3})$.

Además, si $F/k$ es una extensión finita, es claro que $F/k$ es finitamente generada. En efecto, si $\{\alpha_1,\ldots , \alpha_m\}$ es una base de $F$ como $k$-espacio vectorial, es fácil ver que $F=k(\alpha_1,\ldots ,\alpha_m)$. Pero la recíproca no es cierta: es decir, que una extensión sea finitamente generada no implica que sea finita. En efecto, $\QQ(\pi)/\QQ$ es una extensión finitamente generada (simple) pero no es finita, pues si lo fuera, vimos que eso implica que es algebraica, y así $\pi$ sería algebraico. Pero esto es falso.
\end{obs}

\begin{prop}
Sea $F/k$ una extensión. Entonces $F/k$ es finita si y sólo si es finitamente generada y algebraica.
\begin{proof}

$(\Longrightarrow)$ Con una proposición anterior vimos que finita implica algebraica y en la observación anterior vimos que finita implica finitamente generada.

$(\Longleftarrow)$ Supongamos que $F=k(\alpha_1,\ldots , \alpha_m)$ con los $\alpha_i$ algebraicos sobre $k$. Como $\alpha_1$ es algebraico sobre $k$, tenemos que $[k(\alpha_1):k]=n_1<+\infty$. Como $\alpha_2$ es algebraico sobre $k$, lo es sobre $k(\alpha_1)$, y así $[k(\alpha_1,\alpha_2):k(\alpha_1)]=n_2<+\infty$. Continuando así, tenemos que $[k(\alpha_1,\ldots ,\alpha_{i+1}):k(\alpha_1,\ldots , \alpha_i)]=n_{i+1}<+\infty$. Simplemente multiplicando los grados tenemos que $[F:k]=\displaystyle\prod_{i=1}^n n_i < +\infty$.

\end{proof}
\end{prop}

\begin{obs}
Notemos que en la demostración $(\Longleftarrow)$ sólo usamos que los $\alpha_i$ son algebraicos sobre $k$. Entonces, si $\alpha_1,\ldots ,\alpha_m$ son algebraicos sobre $k$ resulta que toda la extensión $k(\alpha_1,\ldots , \alpha_m)/k$ es algebraica. En efecto, si $F/k$ es una extensión y $S\subseteq F$ un subconjunto (ya sea finito o infinito), si para todo $s\in S$ resulta que $s$ es algebraico sobre $k$ entonces $k(S)/k$ resulta algebraica. Esto es porque si $\alpha\in k(S)$ existen $s_1,\ldots , s_m\in S$ tales que $\alpha \in k(s_1,\ldots , s_m)$ y esta es algebraica.
\end{obs}

\section{Clases Distinguidas}
Consideremos una forma muy simple de obtener una extensión de cuerpos nueva dadas $E/k$ y $F/k$.

\begin{defn}
Sean $k,E,F,L$ cuerpos tales que $k\subseteq E,F$ y $E,F\subseteq L$. Se define el \textbf{compuesto} de $E$ y $F$ como el mínimo subcuerpo de $L$ que contiene a $E$ y $F$. Es decir, tenemos el siguiente diagrama de extensiones de cuerpos:
\begin{center}
\begin{tikzcd}[row sep=1.3em,column sep=1.3em,minimum width=2em]
 & L\arrow[dash]{d} & \\
 & EF\arrow[dash]{rd} &  \\
E \arrow[dash]{ur}\arrow[dash]{dr} & & F \\
& k\arrow[dash]{ur} &
\end{tikzcd}
\end{center}
Notar que por definición, se tiene que $EF= E(F)=F(E)$.
\end{defn}

\begin{obs}
Si $E=k(\alpha_1,\ldots , \alpha_m)$ entonces $EF= F(\alpha_1,\ldots , \alpha_m)$ (pues nuevamente, es el cuerpo más chico que contiene a $E$ y $F$).
\end{obs}

\begin{defn}
Una clase $\mathscr{C}$ de extensiones se dice \textbf{distinguida} si satisface:
\begin{itemize}
\item Dada una torre de extensiones $k\subseteq F\subseteq E$, se tiene que $E/F\in\mathscr{C}$ y $F/k\in\mathscr{C}$ si y sólo si $E/k\in \mathscr{C}$.
\item Si $E/k\in\mathscr{C}$ entonces $EF/F\in\mathscr{C}$.
\end{itemize}
\end{defn}

\begin{obs}
Si $\mathscr{C}$ es una clase distinguida y $E/k, F/k\in\mathscr{C}$ entonces debemos tener que $EF/k\in\mathscr{C}$. En efecto, esto es porque por la segunda condición $EF/F\in\mathscr{C}$ y por la primera, como $EF/F, F/k\in\mathscr{C}$ entonces $EF/k\in\mathscr{C}$.
\end{obs}

\begin{prop}
La clase de extensiones finitas es una clase distinguida.
\begin{proof}
Para ver la primera condición, notemos que $[E:k] = [E:F][F:k]$. Entonces, $E/k$ es infinita si y sólo si alguna de las $E/F$ o $F/k$ lo son. Para ver la segunda condición, notemos que como $E/k$ es finita, es finitamente generada y algebraica. Esto quiere decir que $E=k(\alpha_1,\ldots ,\alpha_m)$ y así $EF=F(\alpha_1,\ldots ,\alpha_m)$. Por lo tanto $EF/F$ es finitamente generada y algebraica, pues los $\alpha_i$ son algebraicos sobre $k\subseteq F$ y así sobre $F$. Esto quiere decir que $EF/F$ es finita, como queríamos probar.
\end{proof}
\end{prop}

\begin{prop}
La clase de extensiones algebraicas es una clase distinguida.
\begin{proof}
Para ver la primera condición, notemos que si $E/k$ es algebraica, entonces $E/F$ lo es (pues todo $x\in E$ es raíz de un polinomio con coeficientes en $k\subseteq F$ y así con coeficientes en $F$) y $F/k$ también (pues si todo $x\in E$ es raíz de un polinomio con coeficientes en $k$, con más razón todo $x\in F\subseteq E$ lo es). Ahora, supongamos que $E/F$ y $F/k$ son algebraicas. Sea $\alpha\in E$. Como $\alpha$ es algebraico sobre $F$, existen $a_0,\ldots ,a_n\in F$ tales que $a_0 + a_1 \alpha + \ldots + a_n\alpha^n = 0$, con $a_n\neq 0$. Entonces, $\alpha$ es algebraico sobre $k(a_0,\ldots , a_n)$. Entonces, $k(a_0,\ldots , a_n)(\alpha)$ es una extensión finita (por ser finitamente generada y algebraica) de $k(a_0,\ldots ,a_n)$ y esta a su vez es una extensión finita (por lo mismo). Entonces $k(a_0,\ldots ,a_n,\alpha)$ es una extensión finita sobre $k$ y así $\alpha$ es algebraico sobre $k$.
Para ver la segunda condición, notemos que $EF=F(E)$. Entonces, basta ver que todo $x\in E$ es algebraico sobre $F$. Pero lo son sobre $k\subseteq F$. Y estamos.
\end{proof}
\end{prop}

\section{Morfismos de Extensiones}

\begin{defn}
Si $\sigma:E\to F$ es un morfismo de cuerpos (es decir, un morfismo de anillos entre cuerpos). Si $f\in E[x]$, denotamos $f^\sigma\in F[x]$ al polinomio que se obtiene de aplicar $\sigma$ a los coeficientes de $f$. Es decir, si $f=a_0 + a_1x+ \ldots + a_n x^n\in E[x]$ entonces $f^\sigma = \sigma(a_0) + \sigma(a_1)x + \ldots + \sigma(a_n)x^n\in F[x]$.
\end{defn}

\begin{defn}
Un \textbf{morfismo de extensiones} $\sigma: E/k\to F/k$ es un morfismo de cuerpos $\sigma:E\to F$ tal que $\sigma(x)=x\;\forall x\in k$. Es decir, el siguiente diagrama conmuta: \begin{center}
\begin{tikzcd}[row sep=3.3em,column sep=4em,minimum width=2em]
 E\arrow[]{r}[above, font=\normalsize]{\sigma}& F \\
k\arrow[hookrightarrow]{u}[left,font=\normalsize]{\mathrm{inc}} \arrow[hookrightarrow]{ur}[right, font=\normalsize]{\mathrm{inc}} & \\
\end{tikzcd}
\end{center} Es decir, $\sigma:E\to F$ es un morfismo de $k$-álgebras.
\end{defn}

\begin{obs}
Notemos que si $\sigma:E/k\to F/k$ es un morfismo de extensiones, entonces $\sigma:E\to F$ es un monomorfismo $k$-lineal. En efecto, $\sigma$ es monomorfismo pues $E$ es un cuerpo y así el núcleo debe ser trivial, y es $k$-lineal pues $\sigma(\lambda x)=\sigma(\lambda)\sigma(x)=\lambda \sigma(x)$ para todo $\lambda\in k$.

Ahora, si $[E:k]=[F:k]$ y ambas son finitas, entonces $\sigma$ resulta un isomorfismo de $k$-espacios vectoriales (pero no necesariamente de cuerpos, como se puede ver con $\QQ(\sqrt{2})/\QQ$ y $\QQ(\sqrt{3})/\QQ$). Se deduce entonces que si $\sigma:E/k\to E/k$ es un endomorfismo de extensiones, si $E/k$ es una extensión finita entonces $\sigma$ es un automorfismo.

Además, si $\sigma:E/k\to F/k$ y $f\in k[x]$, entonces $f^\sigma = f$ pues $\left.\sigma\right|_{k}= \id$. También, notemos que si $\sigma:E/k\to F/k$ es un morfismo de extensiones y $f\in E[x]$, si $\alpha\in E$ es raíz de $f$, es claro que $\sigma(\alpha)$ es raíz de $f^\sigma\in F[x]$. En efecto, $f(\alpha)=a_0 + a_1\alpha + \ldots + a_n \alpha^n = 0$ y así $0=\sigma(f(\alpha)) = \sigma(a_0) + \sigma(a_1)\sigma(\alpha)+\ldots + \sigma(a_n)\sigma(\alpha)^n = f^\sigma(\sigma(\alpha))$.

En virtud de estas últimas afirmaciones, es evidente que si $\sigma:E/k\to F/k$ es un morfismo de extensiones y $\alpha\in E$ es raíz de $f\in k[x]$, entonces $\sigma(\alpha)$ es raíz de $f$.
\end{obs}

\begin{prop}
Si $E/k$ es una extensión algebraica, entonces todo endomorfismo $\sigma:E/k\to E/k$ es un automorfismo.
\begin{proof}
Ya observamos que $\sigma$ es un monomorfismo $k$-lineal. Con ver que es un epimorfismo ya estaremos. Sea $\alpha\in E$. Debo ver que $\alpha\in\im \sigma$. Sea $p$ el polinomio minimal de $\alpha\in k$ y sean $\alpha_1,\ldots ,\alpha_m$ las raíces de $p$ en $E$ (son finitas porque $E$ es un cuerpo y hay alguna pues $\alpha\in E$). Consideremos entonces $E'=k(\alpha_1,\ldots, \alpha_m)$. La extensión $E'/k$ es finita pues es algebraica y finitamente generada. Como $\left.\sigma\right|_{k}=\id$ y $\sigma(\alpha_i)=\alpha_j$ para algún $j$ (pues $\sigma(\alpha_i)$ es una raíz en $E$ de $p$), tenemos que $\sigma(E')\subseteq E'$. Entonces, $\sigma$ se restringe (y correstringe) bien a un endomorfismo $\overline{\sigma}:E'/k\to E'/k$. Como $E'/k$ es finita, tenemos que $\overline{\sigma}$ es un automorfismo y así existe un $\gamma\in E'\subseteq E$ tal que $\overline{\sigma}(\gamma) = \alpha$. Y estamos.
\end{proof}
\end{prop}

\begin{prop}[Morfismos que salen de extensiones simples]
Sea $k(\alpha)/k$ una extensión simple y $L/k$ alguna extensión.

Si $\alpha$ es trascendente sobre $k$ y $\sigma:k(\alpha)/k\to L/k$ es un morfismo de extensiones, entonces $\sigma(\alpha)\in L$ es trascendente sobre $k$. Más aún, la asignación $\sigma\mapsto\sigma(\alpha)$ induce una correspondencia biunívoca entre los morfismos de extensiones $\sigma:k(\alpha)/k\to L/k$ y los elementos trascendentes de $L$ sobre $k$.

Si $\alpha$ es algebraico sobre $k$ y $p=m(\alpha,k)$ y $\sigma:k(\alpha)/k\to L/k$ es un morfismo de extensiones, entonces $\sigma(\alpha)$ es raíz de $p$ y la aplicación $\sigma\mapsto \sigma(\alpha)$ induce una correspondencia biunívoca entre los morfismos de extensiones $\sigma:k(\alpha)/k\to L/k$ y las raíces de $p$ en $L$.
\begin{proof}
Para la primera parte, notemos que si $\alpha$ es trascendente, entonces $k[\alpha]\simeq k[x]$. Luego, dado $\beta\in L$ existe un único morfismo de anillos $\sigma_\beta: k[\alpha]\to L$ tal que $\sigma_\beta(\alpha)=\beta$ y $\left.\sigma_\beta\right|_{k}=\id$, en virtud de la propiedad universal del anillo de polinomios. Además, notemos que $\sigma_\beta$ se factoriza a través de $k(\alpha)$ si y sólo si para todo $f(\alpha)\neq 0$ se tiene que $\sigma_\beta(f(\alpha))$ es inversible en $L$. Es decir, $\sigma_\beta(f(\alpha))=f(\beta)$ es distinto de $0$ para todo polinomio $f\in k[x]$ no nulo. Es decir, $\beta$ trascendente.
Es decir, por cada trascendente tengo un morfismo y existe algún morfismo si y sólo si el elemento es trascendente.

Para la segunda parte, ya vimos que si $\sigma:k(\alpha)/k\to L/k$ es un morfismo de extensiones entonces $\sigma(\alpha)$ es raíz de $p$. Resta ver que si $\beta\in L$ es raíz de $p$, entonces existe algún morfismo $\sigma_\beta:k(\alpha)/k\to L/k$ tal que $\sigma_\beta(\alpha)=\beta$. Estamos forzados a definir $\sigma_\beta(f(\alpha))=f(\beta)$ (notemos que esto está definido sobre todo $k(\alpha)$ pues al ser $\alpha$ trascendente sobre $k$ tenemos $k(\alpha)=k[\alpha]$). Veamos que está bien definido. En efecto, debo ver que si $f(\alpha)=g(\alpha)\in k(\alpha)$ entonces debo tener $f(\beta)=g(\beta)$. Pero notemos que $(f-g)(\alpha)=0$ y así $p\mid f-g$. Como $\beta$ es raíz de $p$, debemos tener que $(f-g)(\beta)=0$ y así $f(\beta)=g(\beta)$. La proposición sigue.
\end{proof}
\end{prop}

\section{Clausura Algebraica}

\begin{prop}
Sea $k$ un cuerpo y $f\in k[x]$ un polinomio no constante. Entonces, existe $E/k$ una extensión finita tal que $f$ tiene alguna raíz en $E$.
\begin{proof}
Como $f$ es no constante, existe $p\in k[x]$ irreducible tal que $p\mid f$. Basta probar la proposición entonces para $f=p$ irreducible. Como $p$ es irreducible y $k[x]$ es DIP, consideremos el cuerpo $E=k[x]/\langle p\rangle$. Además, $k$ se mete en $E$ pues $k\hookrightarrow k[x]\stackrel{\pi}{\longrightarrow} k[x]/\langle p\rangle$. Es decir, tenemos $\tau = \pi\circ\mathrm{inc}:k\to k[x]/\langle p\rangle =E$ morfismo (inyectivo pues $k$ es cuerpo). Entonces, $\alpha=\overline{x}\in E$ es raíz de $p\in k[x]$ pues $p(\alpha)=p(\overline{x}) = \overline{p(x)}=\overline{0}$. Pero como $E=k[\alpha]$, tenemos que $[E:k]=\gr p < +\infty$. Como queríamos probar.
\end{proof}
\end{prop}

\begin{obs}
En el sentido formal y estricto de la definición, no conseguimos una extensión finita $E/k$ pues $k$ no es un subcuerpo de $E$. Sin embargo, podemos considerar en general a las extensiones $E/k$ como cuerpos $E$ provistos de un morfismo (inyectivo) $\iota:k\to E$. Veamos que si existe $E'$ cuerpo provisto de un morfismo $\tau:k\to E'$, entonces puedo construir un cuerpo $E$ tal que $k\subseteq E$ que sea isomorfo a $E'$.

Sea $S$ un conjunto tal que $\sharp S = \sharp (E'-\tau(k))$ con $S\cap k=\emptyset$. Tomo $E=k\cup S$ y defino $\overline{\tau}:E\to E'$ por alguna biyección tal que $\left.\overline{\tau}\right|_{k} = \tau$. Le damos vía $\overline{\tau}$ estructura a $E$ de cuerpo, que se restringe bien a la estructura de $k$: $x+y := \overline{\tau}^{-1}(\overline{\tau}(x)+\overline{\tau}(y))$ y $xy:=\overline{\tau}^{-1}(\overline{\tau}(x)\overline{\tau}(y))$. Además, $\overline{\tau}:E\to E'$ resulta un isomorfismo de cuerpos con esta estructura.

\end{obs}

\begin{cor}
Sea $k$ un cuerpo $f_1,\ldots ,f_m\in k[x]$ polinomios no constantes. Entonces, existe una extensión finita $E/k$ que tiene alguna raíz $\alpha_i$ de cada $f_i$ para $1\leq i\leq m$.
\begin{proof}
Construyo, por la proposición anterior, $k=E_0 \subseteq E_1\subseteq \ldots \subseteq E_m = E$ cuerpos tales que $f_i\in E_{i-1}[x]$ tiene alguna raíz en $E_i$ y uso que el grado de las extensiones es multiplicativo en torres.
\end{proof}
\end{cor}

\begin{defn}
Un cuerpo $E$ se dice \textbf{algebraicamente cerrado} si todo polinomio $f\in E[x]$ no constante tiene alguna raíz en $E$.
\end{defn}

\begin{teo}
Sea $E$ un cuerpo. Son equivalentes:
\begin{enumerate}
\item $E$ es algebraicamente cerrado.
\item Todo polinomio no constante $f\in E[x]$ se factoriza linealmente en $E[x]$.
\item Todo polinomio $f\in E[x]$ irreducible tiene grado $1$.
\item $E$ no tiene extensiones algebraicas propias. Es decir, si $F/E$ es algebraica entonces $F=E$.
\end{enumerate}
\begin{proof}
$(1)\Longrightarrow (2)$: Sea $f\in E[x]$ no constante. Entonces, existe $\alpha\in E$ raíz de $f$ por ser $E$ algebraicamente cerrado. Esto quiere decir que $f(x) = (x-\alpha) g(x)$ con $\gr g < \gr f$. Recursivamente, $g$ también tiene una raíz y siguiendo así, factorizo linealmente a $f$.

$(2)\Longrightarrow (3)$: Sea $p\in E[x]$ irreducible. En particular, no es constante. Entonces, como se factoriza linealmente por hipótesis, existe un polinomio de grado $1$ que lo divide. Pero como $p$ es irreducible, debe ser el único factor, y así $p$ tiene grado $1$.

$(3)\Longrightarrow (4)$: Sea $F/E$ una extensión algebraica y $\alpha\in F$. Entonces, existe un polinomio $p\in E[x]$ irreducible tal que $p(\alpha)=0$ (el polinomio minimal de $\alpha$ en $E$). Pero por hipótesis, este polinomio tiene grado $1$, y así $\alpha\in E$.

$(4)\Longrightarrow (1)$: Sea $f\in E[x]$ no constante. Por la proposición anterior, existen una extensión finita $F/E$ y $\alpha\in F$ raíz de $f$. Como $F=E$ por hipótesis, tenemos $\alpha\in F=E$ y así todos los polinomios en $E[x]$ no constantes tienen alguna raíz en $E$.
\end{proof}
\end{teo}

\begin{defn}
Sea $k$ un cuerpo. Una \textbf{clausura algebraica} de $k$ es una extensión de cuerpos $E/k$ tal que $E/k$ es algebraica y $E$ es algebraicamente cerrado.
\end{defn}

\begin{prop}
Sea $E/k$ una extensión algebraica. Si todo $f\in k[x]$ no constante se factoriza linealmente en $E[x]$ entonces $E$ es algebraicamente cerrado.
\begin{proof}
Sea $g\in E[x]$ no constante. Debo ver que existe $\alpha\in E$ raíz de $g$. Por una proposición anterior, existe una extensión $F/E$ finita (en particular algebraica) y $\alpha\in F$ raíz de $g$. Como $F/E$ y $E/k$ son algebraicas, tenemos que $F/k$ lo es. Entonces, existe un polinomio irreducible $p\in k[x]$ tal que $p(\alpha)=0$ (el minimal de $\alpha$ en $k$). En particular, $p$ es no constante. Por lo tanto, $p$ se factoriza linealmente en $E[x]$ por hipótesis, y así resulta que $\alpha\in E$. Como queríamos probar.
\end{proof}
\end{prop}

\begin{prop}
Sea $E/k$ una extensión y sea $F=\{x\in E : x \text{ es algebraico sobre }k\}$. Entonces $F$ es un subcuerpo de $E$ que contiene a $k$.
\begin{proof}
Sean $\alpha,\beta\in F$. Entonces, $k(\alpha,\beta)$ es una extensión algebraica. Como es un cuerpo, y $\alpha,\beta\in k(\alpha,\beta)$ tenemos que las sumas, productos e inversos están en $k(\alpha,\beta)$ y en particular son algebraicos sobre $k$. Entonces están en $F$ y resulta un subcuerpo de $E$. Que contiene a $k$ es obvio pues todos los elementos de $k$ son algebraicos sobre $k$.
\end{proof}
\end{prop}

\begin{defn}
Sea $E/k$ una extensión. Definimos la \textbf{clausura algebraica de $k$ en $E$} al subcuerpo $F=\{x\in E : x\text{ es algebraico sobre }k\}$. Lo denotamos $F=\overline{k}^E$.
\end{defn}

\begin{prop}
Si $E/k$ es una extensión y $E$ es algebraicamente cerrado, entonces $\overline{k}^E$ resulta una clausura algebraica de $k$. Es decir, $\overline{k}^E$ es algebraicamente cerrado (pues esa extensión es algebraica por definición).
\begin{proof}
Sea $F=\overline{k}^E$ por comodidad. Como $F/k$ es una extensión algebraica, basta ver que $F$ es algebraicamente cerrado. Para esto, veamos que todo polinomio $f\in k[x]$ se factoriza linealmente en $F[x]$. Pero esto es obvio, pues como $E$ es algebraicamente cerrado y $f\in k[x]\subseteq E[x]$, tenemos que $f$ se factoriza linealmente en $E[x]$. Por lo tanto, $f = c(x-\alpha_1)\cdots (x-\alpha_m)$ para $\alpha_i\in E$. Como los $\alpha_i$ son algebraicos sobre $k$ (por ser raíces de $f\in k[x]$) debemos tener $\alpha_i\in F$. Por lo tanto, $f$ se factoriza linealemente en $F[x]$. La proposición sigue.
\end{proof}
\end{prop}

\begin{obs}
Esta última proposición nos dice que si queremos encontrar una clausura algebraica de $k$, basta con encontrar un cuerpo algebraicamente cerrado que contenga a $k$. La primera demostración del siguiente teorema está basada en esa idea.
\end{obs}

\begin{teo}
Sea $k$ un cuerpo. Entonces existe $E/k$ clausura algebraica.
\begin{proof}[Demostración de Artin]
Por la proposición anterior, basta encontrar una extensión $E/k$ tal que $E$ es algebraicamente cerrado. Sea $\{f_i\}_{i\in I}$ el conjunto de todos los polinomios no constantes de $k[x]$. Consideremos entonces $A=k[x_i]_{i\in I}$. Si $f_i\in k[x]$, $f_i = a_0 + a_1 x + \ldots + a_n x^n$. Considero $f_i(x_i) = a_0 + a_1 x_i + \ldots + a_n x_i^n \in A$. Sea $J=\langle f_i(x_i)\rangle_{i\in I}\subseteq A$ el ideal generado por esos polinomios. Supongamos que $J=A$. En particular, $1\in J$ y así $1=\displaystyle\sum_{\ell=1}^{m} q_{i_\ell} f_{i_\ell}(x_{i_\ell})$ con $q_{i_\ell}\in A$ para $1\leq \ell\leq m$. Pero sabemos que existe una extensión finita $F/k$ que contiene $\{\alpha_{i_\ell}\}_{\ell=1,\ldots, m}$ raíces en $F$ de $f_{i_\ell}\in k[x]$. Por la propiedad universal del álgebra de polinomios, existe un único morfismo $\mu:A\to F$ de $k$-álgebras que cumple $\mu(x_i)=\begin{cases}0 \text{ si }i\neq i_\ell \;\forall \ell \\ \alpha_{i_\ell} \text{ si }i=i_\ell \end{cases}$. Pero entonces tenemos que $1= \mu(1) = \displaystyle\sum_{\ell=1}^m \mu(q_{i_\ell}) f_{i_\ell}(\alpha_{i_\ell}) = 0$. Absurdo.

Esto nos dice que $J\neq A$. Entonces, existe un ideal maximal $\mathfrak{m}\subseteq A$ que contiene a $J$. Consideremos $E_1 = A/\mathfrak{m}$ cuerpo (por ser cociente por un ideal maximal). Como tenemos que $k\hookrightarrow k[x_i]_{i\in I}=A \stackrel{\pi}{\longrightarrow} A/\mathfrak{m}=E_1$, tenemos que $E_1$ contiene una copia de $k$, y así interpretamos a $E_1$ como una extensión de cuerpos de $k$. Pero en $E_1$, todo $f\in k[x]$ no constante tiene alguna raíz. En efecto, $f = f_i$ para algún $i\in I$ (pues estos son todos los polinomios) y así $f_i(\overline{x_i}) = \overline{f_i(x_i)} = \overline{0}\in E_1$.

Inductivamente, construimos una torre de extensiones $k\subseteq E_1\subseteq \ldots \subseteq E_n\subseteq E_{n+1}\subseteq \ldots$ donde $E_j$ es cuerpo para todo $j\in\NN$ y todo $f\in E_j[x]$ no constante tiene alguna raíz en $E_{j+1}$. Consideremos entonces $E = \displaystyle\bigcup_{j\geq 1} E_j$. Esto es un cuerpo con suma y producto heredado de los $\{E_j\}_{j\in\NN}$. Si $f\in E[x]$ no es constante, entonces $f\in E_n[x]$ para algún $n\in\NN$ y así tiene alguna raíz en $E_{n+1}\subseteq E$. Y estamos.
\end{proof}
\begin{proof}[Demostración de Jacobson]
Primero, notemos que si $E/k$ es una extensión algebraica entonces $\sharp E \leq \max\{\sharp k, \aleph_0\}$. En efecto, si $\mathscr{P}_n = \{f\in k[x] : f\text{ irreducible y mónico de grado } n\}$, y $I_n=\{1,\ldots, n\}$ entonces tenemos que $\sharp E \leq \sharp\displaystyle\bigcup_{n\in \NN} \mathscr{P}_n\times I_n$ puesto que cada polinomio de grado $n$ tiene a lo sumo $n$ raíces. Como $\mathscr{P}_n$ está en biyección con $k^{n}$, la afirmación sigue.

Ahora, sea $\mathscr{C}$ un conjunto con $k\subseteq \mathscr{C}$ tal que $\sharp \mathscr{C} > \max\{\sharp k , \aleph_0\}$. Consideremos el conjunto $S=\{(E,+,\cdot): E\subseteq \mathscr{C} \text{ cuerpo tal que }E/k \text{ es algebraica}\}$. Es claro que $S\neq\emptyset$ pues $k\in S$. Definimos un orden parcial $\leq$ en $S$ por $(E,+,\cdot)\leq (E',+',\cdot')$ si $E\subseteq E'$ es un subcuerpo. Por Lema de Zorn en $S$, hay algún elemento maximal $E$. Afirmo que $E/k$ es una clausura algebraica. En efecto, la extensión $E/k$ es algebraica pues $E\in S$. Veamos que $E$ es algebraicamente cerrado, que es equivalente a ver que no tiene extensiones algebraicas propias. Sea $F/E$ una extensión algebraica. Entonces, $\sharp F= \sharp E\leq \max\{\sharp k,\aleph_0\} < \sharp\mathscr{C}$. Entonces, existe $\varphi:F\to \mathscr{C}$ inyectiva tal que $\left.\varphi\right|_{E}=\id$. Esto quiere decir que $\varphi(F)\subseteq\mathscr{C}$ y tiene estructura de cuerpo heredada de $F$ vía $\varphi$. Entonces, $\varphi(F)$ es una extensión algebraica de $E$ (por serlo $F/E$ y $\varphi$ resultar isomorfismo ya que a la biyección la convertimos en isomorfismo por como le dimos la estructura a $\varphi(F)$). Pero como $E/k$ es algebraica, resulta que $\varphi(F)/k$ es algebraica. Por la maximalidad de $E$ resulta que $\varphi(F)\subseteq E$ y así $F=E$. Como queríamos probar.

\end{proof}
\end{teo}

Ahora, vamos a necesitar generalizar un poco un resultado que ya vimos:

\begin{lem}
Sea $k(\alpha)/k$ una extensión algebraica y $\tau:k\to L$ un morfismo de cuerpos. Entonces, existe una biyección entre $\{\sigma:k(\alpha)\to L \text{ morfismo de cuerpos tales que }\left.\sigma\right|_{k}=\tau\}$ y las raíces de $m(\alpha,k)^\tau$ en $L$.
\begin{proof}
Notemos que $\sigma:k(\alpha)\to L$ tal que $\left.\sigma\right|_{k}=\tau$ queda determinado por su valor $\sigma(\alpha)$. Por otra parte, ya vismo que $\sigma(\alpha)$ debe ser raíz de $m(\alpha,k)^\tau$. Ahora, si $\beta\in L$ es raíz de $m(\alpha,k)^\tau$, definimos $\sigma_\beta:k(\alpha)\to L$ por $\sigma_\beta(f(\alpha)) = f^\tau(\beta)$. Chequeemos que esto está bien definido: si $f(\alpha)=g(\alpha)$, entonces $m(\alpha,k)\mid f-g$. Esto implica que $m(\alpha,k)^\tau \mid f^\tau - g^\tau$ y así $f^\tau(\beta) = g^\tau(\beta)$. Por lo tanto a cada morfismo le asociamos una raíz y a cada raíz un morfismo. Y listo.
\end{proof}
\end{lem}

\begin{teo}
Sea $E/k$ una extensión algebraica, $L$ un cuerpo algebraicamente cerrado y $\tau:k\to L$ un morfismo de cuerpos. Entonces, existe $\sigma:E\to L$ que extiende a $\tau$. Es decir, el siguiente diagrama conmuta: \begin{center}
\begin{tikzcd}[row sep=3.3em,column sep=4em,minimum width=2em]
 E\arrow[dashed]{r}[above, font=\normalsize]{\sigma}& L \\
k\arrow[hookrightarrow]{u}[left,font=\normalsize]{\mathrm{inc}} \arrow[]{ur}[right, font=\normalsize]{\mathrm{\tau}} &
\end{tikzcd}
\end{center}
En particular, si $L/k$ es una extensión con $L$ algebraicamente cerrado, entonces existe un morfismo de extensiones $\sigma:E/k\to L/k$. 
Además, si $E/k$ y $L/k$ son clausuras algebraicas, $\sigma$ resulta un isomorfismo.
\begin{proof}
Consideremos el conjunto de las extensiones parciales $$S=\{(F,\eta) : F/k \text{ subextensión de }E/k, \eta:F\to L \text{ morfismo de extensiones tal que }\left.\eta\right|_{k}=\tau\}$$ El conjunto $S$ está ordenado por $(F,\eta)\leq (F',\eta')$ si $F\subseteq F'$ subcuerpo y $\left.\eta'\right|_{k} = \eta$. Veamos que si $\{(F_i,\eta_i)\}_{i\in I}$ es una cadena entonces tiene cota superior. En efecto, $F=\displaystyle\bigcup_{i\in I} F_i$ es un cuerpo y $\eta:F\to L$ queda definido por $\left.\eta\right|_{F_i}=\eta_i$. Luego, $(F,\eta)$ es cota superior y así por Lema de Zorn existe algún elemento maximal $(\tilde{E},\sigma)\in S$. Bastará probar entonces que $\tilde{E}=E$. Si no fuera así, entonces existe $\alpha\in E$, $\alpha\notin \tilde{E}$. Como $E/k$ es algebraica, $\alpha$ resulta algebraico sobre $k$, y así sobre $\tilde{E}$. Por el lema anterior, puedo extender $\sigma$ a $\tilde{\sigma}:\tilde{E}(\alpha)\to L$ pues al ser $L$ algebraicamente cerrado, $m(\alpha,k)^\sigma$ tiene alguna raíz en $L$ y así hay algún morfismo que lo extiende (pues están en biyección). Esto implicaría que $(\tilde{E}(\alpha),\tilde{\sigma})\in S$, que contradice la maximalidad de $(\tilde{E},\sigma)$. Esto quiere decir que $\tilde{E}=E$ y así tengo una extensión como quería.

Para la segunda afirmación, supongamos que $E/k$ y $L/k$ son clausuras algebraicas. Entonces, existe un $\sigma:E/k\to L/k$ por lo ya visto. Como ya sabemos, resulta automáticamente un monomorfismo. Veamos que $\sigma(E)=L$. Como $E/k$ es una clausura algebraica, entonces $\sigma(E)/k$ lo es (por ser $\sigma$ un isomorfismo con su imagen) y como $L/k$ es algebraica, $L/\sigma(E)$ lo es. Pero entonces $L=\sigma(E)$ puesto que las clausuras algebraicas no tienen extensiones algebraicas propias.
\end{proof}
\end{teo}

\begin{defn}
El teorema anterior nos dice que la clausura algebraica de un cuerpo $k$ es única a menos de isomorfismo. Por lo tanto, notamos $\overline{k}$ a la clausura algebraica de $k$ (a menos de isomorfismo).
\end{defn}

\begin{obs}
El Teorema Fundamental del Álgebra nos dice que $\CC/\RR$ es una clausura algebraica (pues sabemos que esa extensión es finita y así algebraica y el teorema nos dice que $\CC$ es algebraicamente cerrado).
\end{obs}

\section{Extensiones Normales}

\begin{defn}
Sea $k$ un cuerpo y $f\in k[x]$. Dada una extensión $E/k$ decimos que $f$ se \textbf{descompone} en $E$ si $f$ se factoriza linealmente en $E[x]$. Dada una extensión algebraica $E/k$ se dice que $E$ es un \textbf{cuerpo de descomposición} de $f\in k[x]$ si cumple:
\begin{itemize}
\item $f$ se descompone en $E$.
\item $E$ está generado (como $k$-álgebra) por las raíces de $f$ en $E$. Es decir, $E=k(\alpha_1,\ldots ,\alpha_m)$ si $\alpha_i$ son las raíces de $f$ en $E$.
\end{itemize}
\end{defn}

\begin{obs}
Una forma de construir el cuerpo de descomposición de $f\in k[x]$ es tomar una clausura $\overline{k}$ y tomar $E = k(\alpha_1,\ldots , \alpha_m)$ con $\alpha_i$ \textbf{todas} las raíces de $f$ en $\overline{k}$.
\end{obs}

\begin{ex}
$\QQ(\sqrt[4]{2})$ no es el cuerpo de descomposición de $f=x^4 - 2$ pues no tiene a todas las raíces de $f$.

Si $k=\QQ$ y queremos el cuerpo de descomposición de $f=ax^2 + bx+c\in\QQ[x]$, basta tomar $\QQ(\sqrt{b^2 - 4ac})$ (pues si las raíces están, el discriminante debe estar y si el discriminante está, conseguimos las raíces fácilmente).

Si $p$ es primo y $\xi_n$ una raíz $n$-ésima primitiva, entonces $\QQ(\sqrt[n]{p},\xi_n)$ es el cuerpo de descomposición de $f=x^n - p$.
\end{ex}

\begin{defn}
Sea $k$ un cuerpo y $(f_i)_{i\in I}\in k[x]$ una familia de polinomios. Una extensión algebraica $E/k$ se dice un \textbf{cuerpo de descomposición} de la familia $(f_i)_{i\in I}$ si \begin{itemize}\item Todos los $f_i$ se descomponen en $E$.\item $E$ está generado como $k$-álgebra por las raíces en $E$ de todos los $f_i$. \end{itemize}
\end{defn}

\begin{obs}
Dado $k$ y $(f_i)_{i\in I}\in k[x]$ se puede construir un cuerpo de descomposición de la familia $(f_i)_{i\in I}$ tomando una clausura algebraica $\overline{k}$ y $E=k(\alpha_j^{i})_{i\in I, j\in A_i}$ donde $\alpha_j^{i}$ son las raíces de $f_i$ en $\overline{k}$.
\end{obs}

\begin{prop}
Sean $E/k$ y $F/k$ cuerpos de descomposición de la familia $(f_i)_{i\in I}\in k[x]$. Entonces, existe $\sigma:E/k\to F/k$ isomorfismo de extensiones.
\begin{proof}
Tomo $\overline{F}$ una clausura algebraica de $F$. Notemos que existe algún morfismo $\sigma:E/k\to \overline{F}/k$ de extensiones (en virtud del último teorema de la sección anterior, tomando $\tau$ por la inclusión y usando que $\overline{F}$ es algebraicamente cerrado).  Para todo $i\in I$, escribimos $f_i = c_i (x - \alpha_1^i)\cdots (x - \alpha_m^i)$ con $\alpha_j^i\in E$ las raíces de $f_i$. Como $f_i\in k[x]$, entonces tenemos $f_i = f_i^\sigma = c_i (x - \sigma(\alpha_1^i))\cdots (x - \sigma(\alpha_m^i))\in\overline{F}[x]$. Como $\sigma(\alpha_j^i)$ son las raíces de $f_i^\sigma\in \overline{F}[x]$, debemos tener que están en $F$ por ser $F$ un cuerpo de descomposición. Por lo tanto, esto nos dice que $\sigma(E)\subseteq F$. Además, como toda raíz de $f_i$ es $\sigma(\alpha_j^i)$ y así está generado por esas, $\sigma$ es sobreyectivo. La proposición sigue.
\end{proof}
\end{prop}

\begin{obs}
El cuerpo de descomposición de $\{f_1,\ldots ,f_m\}\subseteq k[x]$ es el cuerpo de descomposición de $f=f_1\cdots f_m$. Por lo tanto, o bien estudiamos familias infinitas o sólo un polinomio.
\end{obs}

\begin{teo}
Sea $F/k$ una extensión algebraica y sea $\overline{k}$ una clausura algebraica de $k$ que contiene a $F$. Son equivalentes:
\begin{enumerate}
\item Para todo $\sigma:F/k\to \overline{k}/k$ morfismo de extensiones tenemos que $\sigma(F)=F$.
\item $F$ es el cuerpo de descomposición de una familia de polinomios en $k[x]$.
\item Todo $f\in k[x]$ irreducible que tiene alguna raíz en $F$ se descompone en $F[x]$.
\end{enumerate}
\begin{proof}
$(1)\Longrightarrow (2)$ Sea $\alpha\in F$. Como $F/k$ es una extensión algebraica, $\alpha$ es algebraico sobre $k$. Tomemos $p_\alpha = m(\alpha,k)$ el polinomio minimal de $\alpha$ en $k$. Sea $\beta\in\overline{k}$ raíz de $p_\alpha$. Existe $\mu: k(\alpha)/k\to \overline{k}/k$ tal que $\mu(\alpha)=\beta$. Como $F/k(\alpha)$ es algebraica, puedo extender a $\mu$, por el teorema anterior, a un $\sigma:F\to \overline{k}$. Como $\mu$ fija a $k$ y $\mu(\alpha)=\beta$ entonces $\left.\sigma\right|_{k} = \id$ y $\sigma(\alpha)=\beta$. Esto quiere decir que $\sigma:F/k\to\overline{k}/k$ es un morfismo de extensiones que manda $\alpha$ a $\beta$. Por hipótesis, $\sigma(\alpha)\in F$ y así $\beta\in F$. Entonces, $p_\alpha$ se factoriza linealmente en $F[x]$. Por lo tanto, $F$ es el cuerpo de descomposición de $\{p_{\alpha}\}_{\alpha\in F}$ y la implicación sigue.

$(1)\Longrightarrow (3)$ Tomo $f\in k[x]$ irreducible con $\alpha\in F$ raíz de $f$. Entonces, $f=p_\alpha$ (salvo constante). Por la demostración de $(1)\Longrightarrow (2)$ tenemos que $f$ se descompone en $F$ y listo.

$(2)\Longrightarrow (1)$ Sea $\sigma:F/k\to \overline{k}/k$ un morfismo de extensiones. Sé que $F$ por hipótesis es $F=k(\alpha_j^{i})_{i\in I, j\in A_i}$ con $\alpha_j^i$ raíces de $f_i$ en $\overline{k}$. Pero notemos que $\sigma(\alpha_j^i) = \alpha_{j'}^i$ pues debe ser raíz de $f^\sigma = f$. Entonces, como el morfismo queda determinado por su acción sobre los elementos que agrego, tenemos $\sigma(F)\subseteq F$ y listo.

$(3)\Longrightarrow (1)$ Sea $\sigma:F/k\to \overline{k}/k$ un morfismo de extensiones. Sea $\alpha\in F$. Consideremos $p_\alpha = m(\alpha,F)$ el polinomio minimal de $\alpha$ en $F$. Notemos que $\sigma(\alpha)$ es raíz de $p_\alpha$ en $\overline{k}$. Por hipótesis, como se factoriza linealmente, $\sigma(\alpha)\in F$ y listo.
\end{proof}
\end{teo}

\begin{defn}
Una extensión algebraica que cumpla (cualquiera de) las tres condiciones anteriores se dice \textbf{normal}.
\end{defn}

\begin{obs}
Supongamos que $F=k(\alpha_1,\ldots , \alpha_r)$ con $\alpha_i$ algebraicos sobre $k$. $F/k$ es normal si y sólo si para cada $\alpha_i$ las raíces del polinomio minimal $m(\alpha_i,k)$ en $\overline{k}\supseteq F$ ya están en $F$. En efecto, sabemos que un morfismo de extensiones $\sigma$ queda determinado por su acción en los generadores $\alpha_i$ y sabemos que $\sigma(\alpha_i)\in F$ pues cualquier $\sigma$ manda raíces en raíces que sabemos que están en $F$.
\end{obs}

\begin{prop}
Si $F/k$ es una extensión de grado $2$, entonces $F/k$ es normal.
\begin{proof}
Como $F/k$ es finita, sabemos que es algebraica. Sea $\alpha\in F$, $\alpha\notin k$. Considero $k\subseteq k(\alpha)\subseteq F$. Como $[k(\alpha):k]>1$ pues $\alpha\notin k$, tenemos que $k(\alpha)=F$ pues $2=[F:k] = [F:k(\alpha)][k(\alpha):k]$, y así $[F:k(\alpha)]=1$. Entonces, el polinomio minimal $m(\alpha,k)$ tiene grado $2$. En $\overline{k}$ esto quiere decir que $m(\alpha,k) = (x-\alpha)(x-\beta)$. Como $\alpha$ es raíz de $m(\alpha,k)$ y $\alpha\in k(\alpha)$ entonces $x-\alpha \mid m(\alpha,k)$ en $k(\alpha)[x]$. Pero esto quiere decir que $x-\beta = \dfrac{m(\alpha,k)}{x-\alpha}\in k(\alpha)[x]$. Por lo tanto, $\beta \in k(\alpha)$ y así todas las raíces de $m(\alpha,k)$ están en $F$ y, por la observación anterior, estamos.
\end{proof}
\end{prop}

\begin{ex}
Las extensiones normales no forman una clase distinguida. $\QQ(\sqrt[4]{2})/\QQ$ no es normal pues $\sqrt[4]{2}\, i\notin\QQ(\sqrt[4]{2})$, mientras que $\QQ(\sqrt{2})/\QQ$ y $\QQ(\sqrt[4]{2})/\QQ(\sqrt{2})$ sí lo son por ser extensiones cuadráticas. Esto muestra en particular la no-transitividad de la normalidad de extensiones.
\end{ex}

\begin{prop}
Supongamos que tenemos $k\subseteq E,F \subseteq L$ cuerpos. Entonces, se cumplen las siguientes:
\begin{enumerate}
\item Si $E/k$ es normal entonces $EF/F$ es normal.
\item Si $L/k$ es normal entonces $L/E$ es normal.
\item Si $E/k$ y $F/k$ son normales, entonces $EF/k$ y $E\cap F/k$ son normales.
\end{enumerate}
\begin{proof}
\begin{enumerate}

\item Dado un morfismo de extensiones $\sigma:EF/F\to \overline{EF}/F$ debo ver que $\sigma(EF)\subseteq EF$. Pero $\sigma(EF)\subseteq \sigma(E)\sigma(F)$ trivialmente. Pero $\sigma(E)=E$ por ser $E/k$ una extensión normal (el morfismo se correstringe a un morfismo de extensiones sobre $k$), y $\sigma(F)=F$ pues queda fijo por ser morfismo de extensiones.

\item Si tomo un morfismo $\sigma:L/E \to \overline{L}/E$, quiero ver que $\sigma(L)=L$. Pero ese morfismo se restringe y correstringe a un morfismo $\overline{\sigma}: L/k \to \overline{L}/k$, en el cual sabemos que $\overline{\sigma}(L) = L$ por ser $L/k$ normal. Y listo.

\item Es fácil ver que $EF/k$ haciendo el mismo truco que para el primer ítem. Es decir, si $\sigma: EF/k\to \overline{EF}/k$ es un morfismo de extensiones, tenemos que $\sigma(EF)\subseteq\sigma(E)\sigma(F)$ y tenemos que $\sigma(E)=E$ pues $\sigma$ se restringe a un morfismo de extensiones $E/k\to\overline{EF}/k$ y sabemos que $E/k$ es normal, y que $\sigma(F)=F$ por la razón análoga. Que la intersección es normal es obvio pues para cualquier $f\in k[x]$ irreducible que tiene alguna raíz en $E\cap F$, como $E/k$ es normal, sabemos que $f$ se descompone sobre $E$ y como $F/k$ es normal, $f$ se descompone sobre $F$. Es decir, todas sus raíces están en $E$ y $F$ y así $f$ se descompone en $E\cap F$.

\end{enumerate}
\end{proof}
\end{prop}

\begin{prop}
Sea $k$ un cuerpo, $f\in k[x]$ un polinomio y $F/k$ el cuerpo de descomposición de $f$. Entonces $[F:k]\leq (\gr f)!$.
\begin{proof}
Tomo $\overline{k}$ una clausura algebraica de $k$. Como $F/k$ es el cuerpo de descomposición de $f$, tenemos que $F=k(\alpha_1,\ldots ,\alpha_r)$ donde $\{\alpha_1,\ldots ,\alpha_r\}$ son todas las raíces de $f$ en $\overline{k}$. Como la cantidad de raíces es menor o igual que el grado del polinomio, tenemos que $r\leq \gr f$. Notemos que $[k(\alpha_1):k]\leq n$ ya que $[k(\alpha_1):k]=\gr m(\alpha,k)$ y $m(\alpha,k)\mid f$. Como $\alpha_1\in k(\alpha_1)$ y $f(\alpha_1)=  0$ entonces $x-\alpha_1 \mid f$ en $k(\alpha_1)[x]$. Por lo tanto $\dfrac{f}{x-\alpha_1}\in k(\alpha_1)[x]$ tiene grado $\gr f - 1$. Como $m(\alpha_2, k(\alpha_1))\mid \dfrac{f}{x-\alpha_1}$, tenemos que su grado es menor o igual que $\gr f - 1$. Procediendo inductivamente, podemos probar para cada $i$ que $[k(\alpha_1,\ldots ,\alpha_{i+1}): k(\alpha_1,\ldots ,\alpha_i)]\leq \gr f - i$. Como $r\leq n$, usando que el grado es multiplicativo en torres obtenemos $[k(\alpha_1,\ldots , \alpha_r):k]\leq (\gr f)(\gr f -1)\cdots (\gr f - (r-1))\leq (\gr f)!$. Como queríamos probar.
\end{proof} 
\end{prop}

\section{Raíces múltiples y Separabilidad}

\begin{prop}
Sea $F/k$ una extensión de cuerpos y $f,g\in k[x]$. Si $\mcd_k(f:g)$ denota al máximo común divisor entre $f$ y $g$ en $k[x]$ y $\mcd_F(f,g)$ denota al máximo común divisor entre $f$ y $g$, entonces $\mcd_k(f:g) = \mcd_F(f:g)$, y lo denotamos simplemente $(f:g)$.
\begin{proof}
Notemos que $\mcd_k(f:g)\mid \mcd_F(f:g)$ en $F[x]$ pues $\mcd_k(f:g)$ divide a ambos polinomios en $k[x]$ y así en $F[x]$, y por lo tanto tiene que dividir máximo común divisor de $f,g$ en $F[x]$. Ahora, notemos que como $k[x]$ es un dominio de ideales principales, $\langle f,g\rangle = \langle \mcd_k(f:g)\rangle$, y así existen $a,b\in k[x]$ tales que $\mcd_k(f:g) = af + bg$. En particular, esta igualdad también vale en $F[x]$. Pero como $\mcd_F(f:g)\mid f,g$ en $F[x]$, tenemos que $\mcd_F(f:g)\mid af + bg = \mcd_k(f:g)$. Como los máximos comunes divisores de polinomios son mónicos por definición y logramos probar que se dividen mutuamente, deben ser lo mismo. Y estamos.
\end{proof}
\end{prop}

\begin{obs}
En particular, si $f,g\in k[x]$ son irreducibles y mónicos ($f\neq g$) sabemos que deben ser coprimos en $k[x]$. Pero la proposición anterior nos dice que deben ser coprimos en $\overline{k}[x]$. Es decir, no tienen ninguna raíz en común (en $\overline{k}$).
\end{obs}

\begin{defn}
Sea $f\in k[x]$ no constante. Podemos factorizar al polinomio $f$ (en $\overline{k}$) como $f=c(x-\alpha_1)^{m_1}\cdots (x-\alpha_r)^{m_r}$ con $\alpha_i\neq \alpha_j$ si $i\neq j$. Decimos que $\alpha_i$ tiene \textbf{multiplicidad} $m_i$ en $f$. Se dice que $\alpha_i$ es una raíz múltiple si $m_i> 1$ y es simple si $\alpha_i=1$.
\end{defn}

\begin{defn}
Sea $f\in k[x]$, $f=a_0 + a_1 x + \ldots + a_n x^n$. Se define el polinomio derivado de $f$ por $f'=a_1 + 2a_2 x + \ldots + na_nx^n$.
\end{defn}

\begin{obs}
Es fácil ver que $(f+g)' = f' + g'$, $(fg)' = f'g + fg'$ y que $(af)' = af'$ ($a\in k$).
\end{obs}

\begin{prop}
Sea $f\in F[x]$ y $\alpha\in F$ raíz de $f$. Son equivalentes:
\begin{enumerate}
\item $\alpha$ es raíz múltiple (es decir, $(x-\alpha)^2 \mid f$ en $F[x]$).
\item $x-\alpha\mid f$ y $x-\alpha\mid f'$ en $F[x]$.
\end{enumerate}
\begin{proof}
$(1)\Longrightarrow (2)$: Supongamos que $f= (x-\alpha)^2 g$. Entonces, por la regla del producto, $f' = 2(x-\alpha)g + (x-\alpha)^2 g'$. En particular, $x-\alpha\mid f'$.

$(2)\Longrightarrow (1)$: Como $x-\alpha \mid f$ tenemos que $f = (x-\alpha)h$ y así $f' = (x-\alpha)h' + h$. Pero como $x-\alpha\mid f'$ tenemos que $x-\alpha \mid h$ y así $ h = (x-\alpha)\tilde{h}$ y así $f=(x-\alpha)^2 \tilde{h}$.
\end{proof}
\end{prop}

\begin{cor}
Sea $f\in k[x]$. $f$ no tiene raíces múltiples (en $\overline{k}$) si y sólo si $(f:f')=1$.
\begin{proof}
$(\Longrightarrow)$ Si $(f:f')\neq 1$ entonces en $\overline{k}[x]$ no serían coprimos tampoco. Entonces, existe un polinomio irreducible en $\overline{k}[x]$ que divide a $f$ y $f'$. Pero los irreducibles en un cuerpo algebraicamente cerrados son los polinomios de grado $1$. Esto quiere decir que existe un polinomio $x-\alpha \mid f,f'$ y así $\alpha$ sería raíz múltiple de $f$.

$(\Longleftarrow)$ Si $f$ tiene raíz múltiple $\alpha\in\overline{k}$ entonces $x-\alpha \mid f,f'$. Entonces, $(f:f')\neq 1$.
\end{proof}
\end{cor}

\begin{defn}
Un polinomio $f\in k[x]$ se dice \textbf{separable} si $f\neq 0$ y sus raíces (en $\overline{k}$) son todas simples.
\end{defn}

\begin{prop}
Sea $f\in k[x]$ irreducible. Son equivalentes:
\begin{enumerate}
\item $f$ es separable.
\item $f$ tiene alguna raíz simple (en $\overline{k}$).
\item $f'\neq 0$.
\end{enumerate}
\begin{proof}
$(1)\Longrightarrow(2)$ No hay nada que probar.

$(2)\Longrightarrow(3)$ Sea $\alpha\in\overline{k}$ una raíz simple de $f$. Si $f' = 0$ entonces $x-\alpha\mid f$ y $x-\alpha\mid f'=0$. Entonces $\alpha$ resultaría raíz múltiple. Absurdo.

$(3)\Longrightarrow(1)$ Si $f'\neq 0$, como $\gr f'< \gr f$ y $f$ irreducible, tenemos que $f\nmid f'$. Por lo tanto, $(f:f')=1$. Entonces $f$ no tiene raíces múltiples y listo.
\end{proof}
\end{prop}

\begin{cor}
Si $\car k = 0$ entonces todo polinomio irreducible es separable.
\begin{proof}
Si $a_n$ es el coeficiente principal de $f$, entonces $na_n\neq 0$ es el coeficiente principal de $f'$ y así $f'\neq 0$.
\end{proof}
\end{cor}

\begin{ex}
Puede pasar que algún polinomio irreducible en característica $p$ tenga derivada nula. Es decir, no todo polinomio irreducible es separable. En efecto, supongamos que $k$ es un cuerpo con $\car k = p$. Sea $a\in k$ tal que no existe ningún $b\in k$ con $b^p = a$. Por ejemplo, consideremos $k=\FF_p(t)$ con $t$ trascendente y $a=t$. Sea $f = x^p - a\in \FF_p(t)[x]$. En $\overline{\FF_p(t)}$ existe algún $\alpha$ raíz de $f$. Tenemos entonces $\alpha^p = a$. Es decir, $x^p - a = x^p - \alpha^p = (x-\alpha)^p$. Por lo tanto, $\alpha$ es la única raíz de $\overline{\FF_p(t)}$ y tiene multiplicidad $p$. Además, $f$ es irreducible pues como $\alpha\notin \FF_p(t)$, tenemos que $\alpha^r \notin \FF_p(t)$ para todo $1\leq r \leq p-1$. Entonces, si $f=gh$ tenemos que $g= (x-\alpha)^r\in\overline{\FF_p(t)}[x]$ por la unicidad de la factorización. Por lo tanto, el coeficiente independiente de $g$ es $\alpha^r$ que no está en $\FF_p(t)$.

Entonces, $f=x^p - a\in \FF_p(t)[x]$ es irreducible, pero $f' = 0$.

Si $\car k = p$, entonces $f'=0$ si y sólo si $ra_r = 0$ en $k$. Cuando $p\nmid r$ tiene que pasar que $a_r=0$. Entonces, $f'=0$ si y sólo si $f = a_0 + a_p x^p + \ldots + a_{rp}x^{rp} = g(x^p)$ donde  $g = a_0 + a_p x + \ldots + a_{rp}x^r$.
\end{ex}

\begin{defn}
Un cuerpo $k$ se dice \textbf{perfecto} si todo $f\in k[x]$ irreducible es separable.
\end{defn}

\begin{obs}
Todo cuerpo de característica $0$ es perfecto.
\end{obs}

\begin{prop}
Sea $k$ un cuerpo de característica $p>0$. $k$ es perfecto si y sólo si $k=k^p$ (es decir, para todo $a\in k$ existe $b\in k$ tal que $a=b^p$).
\begin{proof}
$(\Longrightarrow)$ Si $k\neq k^p$ entonces $f=x^p - a$ es irreducible con $a\neq b^p$, con el mismo razonamiento que en el ejemplo anterior. Pero tenemos $f'=0$. Es decir, es irreducible pero no separable.

$(\Longleftarrow)$ Sea $f\in k[x]$ irreducible. Debo ver que $f'\neq 0$. Supongamos que $f'=0$. Entonces $f(x)=g(x^p)$ para $g\in k[x]$. Es decir, $f = a_0 + a_1x^p + \ldots + a_r x^{rp}$. Como $k=k^p$, tenemos que $a_i = b_i^p$ para ciertos $b_i\in k$. Pero entonces tenemos que $$f= b_0^p + \ldots + b_r^p x^{rp} = (b_0 + b_1x + \ldots + b_rx^r)^p$$ que no es irreducible. La proposición sigue.
\end{proof}
\end{prop}

\begin{prop}
Sea $k$ un cuerpo con característica $p>0$ y $f\in k[x]$ irreducible. Entonces, existe $q\in k[x]$ irreducible y separable y $h\in\NN_0$ tal que $f(x) = q(x^{p^h})$.
\begin{proof}
Consideremos $A=\{n\in\NN_0 : \exists g\in k[x] \text{ con }f=g(x^{p^n})\}$. Notemos que $0\in A$ y así $A\neq\emptyset$ y que está acotado pues $\gr f<+\infty$. Sea $h$ el máximo de $A$. Sea $q\in k[x]$ tal que $f(x)=q(x^{p^h})$. Es claro que $q$ es irreducible pues si $q=q_1q_2$ con $q_1,q_2\in k[x]$ tendríamos que $f(x) = q_1(x^{p^h})q_2(x^{p^h})$, que es absurdo pues $f$ es irreducible. Además, $q$ es separable pues si no lo fuere tendríamos que $q'=0$ y así $q(x)=g(x^p)$. Esto implica que $f(x)=g(x^{p^{h+1}})$, que contradice la maximalidad de $h$.
\end{proof}
\end{prop}

\begin{cor}
Si $\car k = p>0$ y $\alpha\in\overline{k}$ entonces las raíces de $m(\alpha,k)$ (en $\overline{k}$) tienen todas la misma multiplicidad y esa multiplicidad es una potencia de $p$.
\begin{proof}
Por la proposición anterior, existe $q\in k[x]$ irreducible y separable tal que $m(\alpha,k) = q(x^{p^h})$ para algún $h\in\NN_0$. Sean $y_1,\ldots , y_n$ las raíces de $q$ en $\overline{k}$. Tenemos que $y_i\neq y_j$ si $i\neq j$ ya que $q$ es separable. Es decir, $q=\displaystyle\prod_{i=1}^n (x-y_i)\in\overline{k}[x]$. Sean $x_1,\ldots , x_n\in\overline{k}$ tales que $x_i^{p^h} = y_i$. Es claro que $x_i\neq x_j$ si $i\neq j$ pues si no al elevar a la $p^h$ tendríamos $y_i=y_j$. Entonces, $m(\alpha,k) = q(x^{p^h}) = \displaystyle\prod_{i=1}^n (x^{p^h} - y_i) = \prod_{i=1}^n (x^{p^h} - x_i^{p^h}) = \prod_{i=1}^n (x-x_i)^{p^h}$. Esto quiere decir que todas las raíces del minimal de $\alpha$ en $k$ tienen multiplicidad $p^h$ como queríamos probar.
\end{proof}
\end{cor}

\begin{defn}
Sean $E/k, F/k$ son extensiones de cuerpos. Denotamos al conjunto de morfismos de extensiones por $\hom(E/k,F/k) = \{\sigma:E/k\to F/k | \sigma \text{ morfismo de extensiones} \}$, a los endomorfismos $\End(E/k) = \hom(E/k,E/k)$ y a los automorfismos (endomorfismos que son isomorfismos) $\Aut(E/k)$.
\end{defn}

\begin{defn}
Sea $E/k$ una extensión algebraica. El \textbf{grado de separabilidad} de $E/k$ es $[E:k]_s = \sharp \hom(E/k,\overline{k}/k)$. Notemos que la definición no depende de la clausura algebraica elegida pues es única a menos de isomorfismo.
\end{defn}

\begin{obs}
Si $\alpha$ es algebraico sobre $k$, entonces $[k(\alpha):k]_s$ es la cantidad de raíces distintas de $m(\alpha,k)$ (en $\overline{k}$), pues vimos que tenemos una biyección entre los morfismos de extensiones $\sigma:k(\alpha)/k\to\overline{k}/k$ y las raíces del minimal de $\alpha$ en $k$. Entonces, es claro que $[k(\alpha):k]_s \leq [k(\alpha):k]$, y que $m(\alpha,k)$ es separable si y sólo si $[k(\alpha):k]_s=[k(\alpha):k]$.
\end{obs}

\begin{prop}
Sea $k\subseteq F\subseteq E$ una torre de extensiones finitas. Entonces el grado de separabilidad es multiplicativo en torres. Es decir, $[E:k]_s = [E:F]_s [F:k]_s$.
\begin{proof}
Ejercicio.
\end{proof}
\end{prop}

\begin{cor}
Si $E/k$ es una extensión finita, entonces $[E:k]_s \leq [E:k]$.
\begin{proof}
Notemos que, por la observación anterior, para cada $1\leq i\leq r$ tenemos que $[k(\alpha_1,\ldots ,\alpha_{i-1})(\alpha_i) : k(\alpha_1,\ldots , \alpha_{i-1})]_s \leq [k(\alpha_1,\ldots ,\alpha_{i-1})(\alpha_i) : k(\alpha_1,\ldots , \alpha_{i-1})]$ y ahora simplemente es usar que ambos grados son multiplicativos en torres.
\end{proof}
\end{cor}

\begin{defn}
Sea $E/k$ una extensión finita. Decimos que la extensión $E/k$ es \textbf{separable} si $[E:k]_s = [E:k]$. Si $\alpha$ es algebraico sobre $k$ se dice que es separable sobre $k$ si la extensión $k(\alpha)/k$ es separable.
\end{defn}

\begin{obs}
Notemos que $\alpha$ es separable si y sólo si $[k(\alpha):k]_s = [k(\alpha):k]$. Pero eso es que la cantidad de raíces distintas del minimal de $\alpha$ sea exactamente el grado. Es decir, que el minimal de $\alpha$ en $k$ sea separable.
\end{obs}

\begin{obs}
Si tenemos una torre de extensiones finitas $k\subseteq F\subseteq E$, entonces $E/k$ es separable si y sólo si $E/F$ y $F/k$ lo son. En efecto, simplemente es usar que el grado de separabilidad es multiplicativo en torres y que si alguno de los grados de separabilidad fuera más chico que el grado de la extensión entonces el producto quedaría menor también.
\end{obs}

\begin{teo}
Sea $E/k$ una extensión finita. Entonces, $E/k$ es separable si y sólo si para todo $\alpha\in E$ tenemos que $\alpha$ es separable.
\begin{proof}
$(\Longrightarrow)$ Si $\alpha\in E$, entonces tenemos la torre $k\subseteq k(\alpha)\subseteq E$ y como $E/k$ es separable, por la observación anterior, resulta que $k(\alpha)/k$ es separable y así $\alpha$ separable sobre $k$.

$(\Longleftarrow)$ Como $E/k$ es finita, es finitamente generada y algebraica. Luego $E=k(\alpha_1,\ldots ,\alpha_r)$ con cada $\alpha_i$ algebraico sobre $k$. Por hipótesis, cada $\alpha_i$ es separable sobre $k$. Notemos que si $\alpha_i$ es separable sobre $k$ entonces $\alpha_i$ es separable sobre $k(\alpha_1,\ldots , \alpha_{i-1})$. En efecto, esto es porque $m(\alpha_i, k(\alpha_1,\ldots ,\alpha_{i-1}))\mid m(\alpha_i,k)$ y como $m(\alpha_i,k)$ tiene sus raíces simples por ser $\alpha_i$ separable sobre $k$, sus divisores también tienen sus raíces simples. Ahora, como tenemos que cada subextensión en la torre es separable, debemos tener que toda la extensión es separable. Y listo.
\end{proof}
\end{teo}

\begin{obs}
En la demostración anterior probamos que si los generadores son separables entonces la extensión lo es.
\end{obs}

\begin{defn}
Sea $E/k$ una extensión algebraica. Se dice que $E/k$ es \textbf{separable} si para todo $\alpha\in E$ tenemos que $\alpha$ es separable sobre $k$.
\end{defn}

\begin{obs}
Sigue valiendo lo de los generadores: es decir, si $S$ es un conjunto de generadores de $E/k$ y todos los elementos de $S$ son separables sobre $k$ entonces $E/k$ es separable. En efecto, para cada $x\in E$ existen $s_1,\ldots , s_n\in S$ tales que $x\in k(s_1,\ldots , s_n)$. Como los generadores de esa extensión finita son separables, la extensión lo es y así $x$ es separable sobre $k$.
\end{obs}

\begin{prop}
Las extensiones separables forman una clase distinguida.
\begin{proof}
Ejercicio.
\end{proof}
\end{prop}

\begin{obs}
Sea $\alpha$ algebraico sobre $k$. Si $\car k=0$ entonces $\alpha$ es separable sobre $k$ y así $[k(\alpha):k]_s = [k(\alpha):k]$. Ahora, si $\car k = p>0$, entonces existe $h\in\NN_0$ tal que $p^h\cdot [k(\alpha):k]_s = [k(\alpha):k]$, por el corolario que vimos que nos decía que el en el minimal todas las raíces tenían la misma multiplicidad que era una potencia de $p$.
\end{obs}

\begin{prop}
Si $E/k$ es una extensión finita, entonces:
\begin{enumerate}
\item Si $\car k =0$ entonces $[E:k]_s = [E:k]$.
\item Si $\car k = p>0$ entonces $[E:k]_s \mid [E:k]$ y el cociente es una potencia de $p$.
\end{enumerate}
\begin{proof}
Simplemente es usar la observación anterior y que ambos grados son multiplicativos en torre.
\end{proof}
\end{prop}

\begin{defn}
Sea $E/k$ una extensión finita. Se define el \textbf{grado de inseparabilidad} como $$[E:k]_{\mathrm{ins}} = \dfrac{[E:k]}{[E:k]_s}$$
\end{defn}

\begin{obs}
Si $\car k =0$ entonces $[E:k]_{\mathrm{ins}}=1$. Si $\car k =p>0$ entonces $[E:k]_{\mathrm{ins}} = p^n$ con $n\in\NN_0$. Además, si $k\subseteq F\subseteq E$ es una torre de extensiones finitas, tenemos que $[E:k]_{\mathrm{ins}} = [E:F]_{\mathrm{ins}} [F:k]_{\mathrm{ins}}$, pues los otros dos grados son multiplicativos en torres.
\end{obs}

\begin{prop}
Sea $k\subseteq F\subseteq E$ una torre de extensiones algebraicas. Si $F/k$ es separable y $\alpha\in E$ separable sobre $F$, entonces $\alpha$ es separable sobre $k$.
\begin{proof}
Sea $m(\alpha,F)\in F[x]$ el polinomio minimal y $a_0,\ldots , a_n$ sus coeficientes. Consideremos $k(a_0,\ldots , a_n)$, que es algebraica (pues al ser todos los $a_i$ elementos de $F$ que es una extensión algebraica, son algebraicos) y finitamente generada. Entonces, $k(a_0,\ldots , a_n)/k$ es finita y separable (es separable pues el minimal de $\alpha$ en $F$ y en $k(a_0,\ldots , a_n)$ es el mismo). Como $k(a_0,\ldots ,a_n)(\alpha)/k(a_0,\ldots , a_n)$ es separable, tenemos que $k(a_0,\ldots , a_n,\alpha)/k$ es separable y así $\alpha$ es separable sobre $k$. Y estamos.
\end{proof}
\end{prop}

\begin{prop}
Sea $E/k$ una extensión algebraica. Consideremos el subconjunto de $E$ de los separables $F=\{x\in E : x\text{ es separable sobre }k\}$. Entonces:
\begin{enumerate}
\item $F/k$ es una subextensión de $E/k$.
\item $F/k$ es separable.
\item Si $\alpha\in E$ es separable sobre $F$, entonces $\alpha\in F$.
\item Si $E/k$ normal, entonces $F/k$ es normal.	
\end{enumerate}
\begin{proof}
La primera afirmación se sigue de que si $\alpha,\beta\in F$, entonces $k(\alpha,\beta)$ es una extensión separable (pues los generadores lo son), y así todo elemento de la extensión se separable. En particular, $\alpha+\beta, \alpha\beta, \alpha^{-1},\beta^{-1}$. La segunda afirmación es evidente pues todo elemento es separable. La tercera afirmación sigue directamente de la proposición anterior. 

Para la última afirmación, sea $\sigma:F/k\to \overline{k}/k$ un morfismo de extensiones. Debo ver que $\sigma(F)=F$. Como $E/F$ es algebraica, $\sigma$ se extiende $\tilde{\sigma}:E/k\to \overline{k}/k$. Además, como $E/k$ es normal, tenemos que $\tilde{\sigma}(E)=E$. Sea $x\in F$. Quiero ver que $\sigma(x)\in F$. Pero $\sigma(x)=\tilde{\sigma}(x)\in E$. Como $\sigma(x)$ es raíz del minimal de $x$ en $k$ tenemos que $\sigma(x)$ es separable (pues $x$ es separable si y sólo si $m(x,k)$ lo es, y $m(\sigma(x),k)=m(x,k)$). Entonces, $\sigma(x)$ es separable y así está en $F$. La proposición sigue.
\end{proof}
\end{prop}

\begin{defn}
Sea $E/k$ una extensión algebraica. Si miramos el conjunto $$F=\{x\in E : x\text{ es separable sobre }k\}$$ Entonces $F/k$ se llama la \textbf{clausura separable} de $E/k$.
\end{defn}

\begin{defn}
Sea $E/k$ una extensión algebraica y $\car k= p > 0$. Decimos que $\alpha\in E$ es \textbf{puramente inseparable} sobre $k$ si existe un $n\in\NN_0$ tal que $\alpha^{p^n}\in k$.
\end{defn}

\begin{teo}
Sea $E/k$ una extensión algebraica y $\car k = p >0$. Son equivalentes:
\begin{enumerate}
\item $[E:k]_s = 1$.
\item Todo $\alpha\in\overline{k}$ es puramente inseparable sobre $k$.
\item Para todo $\alpha\in E$, $m(\alpha,k) = x^{p^n}-a\in k[x]$ para algún $n\in\NN_0$.
\item $E=k(S)$ donde $S$ es un conjunto de elementos puramente inseparables sobre $k$.
\end{enumerate}
\begin{proof}
$(1)\Longrightarrow (2)$: Sea $\alpha\in E$. Debo ver que $\alpha^{p^n}\in k$ para algún $n\in\NN_0$. Como $[E:k]_s=1$, entonces $[k(\alpha):k]_s=1$ (pues el grado de separabilidad es multiplicativo en torres y $k(\alpha)$ es un subcuerpo de $E$). Esto nos dice que $m(\alpha,k)$ tiene una única raíz en $\overline{k}$. Entonces, $m(\alpha,k) = (x-\alpha)^r$. Pero vimos que las raíces de los minimales tienen multiplicidad una potencia de la característica del cuerpo. Es decir, $m(\alpha,k) = (x-\alpha)^{p^n} = x^{p^n} - \alpha^{p^n}\in k[x]$. Entonces, $\alpha^{p^n}\in k$ como queríamos.

$(2)\Longrightarrow (3)$: Sea $\alpha\in E$ puramente inseparable. Entonces, existe un $n\in\NN_0$ tal que $\alpha^{p^n}\in k$. Esto quiere decir que $\alpha$ es raíz de $x^{p^n} - \alpha^{p^n}\in k[x]$. Pero eso es $(x-\alpha)^{p^n}$. Entonces, $m(\alpha,k)\mid (x-\alpha)^{p^n}$ y así $m(\alpha,k) = (x-\alpha)^r\in k[x]$. Pero vimos que las potencias de los minimales en $\car k = p>0$ es potencia de $p$, así que $m(\alpha,k) = (x-\alpha)^{p^h} = x^{p^h} - \alpha^{p^h}$.

$(3)\Longrightarrow (4)$: Es obvio, ya que todo elemento es puramente inseparable pues tenemos $m(\alpha,k)=x^{p^n}-a$ y así $\alpha^{p^n}= a\in k$.

$(4)\Longrightarrow (1)$: Sea $E=k(\alpha_i)_{i\in I}$ con cada $\alpha_i$ puramente inseparable sobre $k$. Debo ver que $\sharp \hom(E/k,\overline{k}/k=1$. Si tomo la clausura con $E\subseteq \overline{k}$, entonces tenemos que ver que el único morfismo de extensiones $\sigma:E/k\to\overline{k}/k$ entonces $\sigma(\alpha_i) = \alpha_i$ (es decir, que el único morfismo que hay es la inclusión). Pero como $\alpha_i^{p^{h_i}}\in k$, tenemos que $m(\alpha_i,k) = (x-\alpha_i)^{p^{h_i}}$ y así $\alpha_i$ es la única raíz de su minimal. Como $\sigma$ manda raíces del minimal en raíces del minimal, debe ser $\sigma(\alpha_i)=\alpha_i$, como queríamos.
\end{proof}
\end{teo}

\begin{defn}
Una extensión algebraica $E/k$ se dice puramente inseparable si cumple alguna de las propiedades del teorema anterior.
\end{defn}

\begin{prop}
Sea $E/k$ una extensión algebraica y $F$ la clausura separable de $E/k$. Entonces, $E/F$ es puramente inseparable.
\begin{proof}
Sea $\alpha\in E$. Sabemos que existen $n\in\NN_0$ y $g\in F[x]$ irreducible y separable tal que $m(\alpha,F) = g(x^{p^n})$. Sean $r_1,\ldots , r_s$ las raíces de $g$ en una clausura algebraica. Entonces, como $g$ es separable, $g=(x-r_1)\cdots (x-r_s)$. Pero notemos que esto quiere decir que $g(\alpha^{p^n}) = (\alpha^{p^n} - r_1)\cdots (\alpha^{p^n}-r_s) = 0$ y así $\alpha^{p^n} = r_i$ para algún $1\leq i\leq s$. Como $g$ es irreducible, es el minimal de $r_i$ y así, como $g$ es separable, $r_i$ resulta separable. Pero como $r_i\in E$ es separable, por definición $r_i\in F$. Entonces $\alpha^{p^n} = r_i\in F$. Y listo.
\end{proof}
\end{prop}

\section{Extensiones Galoisianas}

\begin{defn}
Una extensión algebraica $E/k$ se dice \textbf{Galoisiana} (o de Galois) si es normal y separable.
\end{defn}

\begin{defn}
Sea $E$ un cuerpo y $G\subseteq \Aut(E)$ un subgrupo de los automorfismos (de cuerpo) de $E$. El \textbf{cuerpo fijo} por $G$ de $E$ se define como $E^G = \{x\in E : \sigma(x)=x \;\forall\sigma\in G\}$. Es fácil ver que $E^G \subseteq E$ es un subcuerpo.
\end{defn}

\begin{defn}
Si la extensión $E/k$ es Galois, entonces $(\Aut(E/k),\circ)$ los automorfismos de extensiones forman un grupo, que llamamos el \textbf{grupo de Galois} de $E/k$ y se lo denota $\Gal(E/k)$.
\end{defn}

\begin{teo}
Si $E/k$ es Galois, entonces $k = E^{\Gal(E/k)}$. Es decir, los elementos fijos dejados por $\Gal(E/k)$ son precisamente los de $k$.
\begin{proof}
La contención $k\subseteq E^{\Gal(E/k)}$ es obvia, pues por definición los automorfismos de $E/k$ dejan fijo a $k$. Para probar la otra contención, sea $\alpha\in E^G$. Es decir, $\alpha\in E$ y $\sigma(\alpha)=\alpha$ para todo $\sigma\in \Gal(E/k)$. Supongamos que $\alpha\notin k$. Esto implica que $\gr m(\alpha,k)>1$. Como $E/k$ es separable, existe un $\beta\in\overline{k}$ tal que $\beta$ es raíz de $m(\alpha,k)$ y $\beta\neq \alpha$.

Pero notemos que podemos definir $\sigma:k(\alpha)/k\to \overline{k}/k$ por $\sigma(\alpha)=\beta$ y $\left. \sigma\right|_{k}=\id$ (por la propiedad universal del anillo de polinomios). Como $E/k(\alpha)$ es algebraica, $\sigma$ se extiende a $\overline{\sigma}: E/k\to \overline{k}/k$ y $\overline{\sigma}(\alpha)=\beta$. Pero como $E/k$ es una extensión normal, tenemos que $\overline{\sigma}(E) = E$. Es decir, $\overline{\sigma}$ es un automorfismo de $E/k$, y así debería dejar fijo a $\alpha$. Absurdo.
\end{proof}
\end{teo}

\begin{obs}
Sea $k\subseteq F\subseteq E$ una torre de extensiones. Si $E/k$ es Galois, entonces $E/F$ es de Galois (pues si $E/k$ es normal, $E/F$ lo es y lo mismo para separable).
\end{obs}

\begin{cor}[Teorema Fundamental de la Teoría de Galois I]
Sea $E/k$ una extensión de Galois. Entonces, la función $$ \varphi:\{\text{Cuerpos intermedios de }E/k\}\to \{\text{Subgrupos de }\Gal(E/k)\}$$ dada por $\varphi(F)=\Gal(E/F)$ es inyectiva.
\begin{proof}
Notemos que $\varphi$ está bien definida ya que $E/F$ es Galois por serlo $E/k$. Además, $\Gal(E/F)\subseteq \Gal(E/k)$ es un subgrupo. Ahora, supongamos que $\varphi(F)=\varphi(F')$ para ciertos $F,F'$ cuerpos intermedios. Entonces, tendríamos que $\Gal(E/F) = \Gal(E/F')$ y así $F = E^{\Gal(E/F)} = E^{\Gal(E/F')} = F'$. Como queríamos probar.
\end{proof}
\end{cor}

\begin{obs}
Si $k\subseteq F\subseteq F'\subseteq E$, entonces $\varphi(F)\supseteq \varphi(F')$ (es decir, es contravariante). Además, si $k\subseteq F,F'\subseteq E$ se tiene que $\Gal(E/FF') = \Gal(E/F)\cap \Gal(E/F')$. Es decir, $\varphi(FF') = \varphi(F)\cap \varphi(F')$.
\end{obs}

\begin{prop}
Sea $E/k$ una extensión finita. Entonces, $E/k$ es Galois si y sólo si $|\Gal(E/k)| = [E:k]$.
\begin{proof}
Si la extensión es finita y Galois, entonces $$|\Gal(E/k)| \stackrel{(1)}{=} \sharp\End(E/k) \stackrel{(2)}{=} \sharp \hom(E/k,\overline{k}/k) \stackrel{(3)}{=} [E:k]_s \stackrel{(4)}{=} [E:k]$$ Donde $(1)$ se da porque la extensión es finita (y así los automorfismos coinciden con los endomorfismos), $(2)$ se da por ser $E/k$ normal, $(3)$ por definición y $(4)$ por ser separable. Para ver la vuelta, si no valiera alguna de las propiedades de ser normal o separable, en la cadena de igualdades anterior habría una desigualdad y no tendríamos que $|\Gal(E/k)|= [E:k]$.
\end{proof}
\end{prop}

\begin{obs}
En particular, si $E/k$ es Galois y finita, entonces el grupo de galois $\Gal(E/k)$ es finito y así la cantidad de subgrupos del grupo de galois son finitos. Por la primera parte del Teorema Fundamental de la Teoría de Galois, sólo existen finitos cuerpos intermedios.
\end{obs}

\begin{prop}
Sea $E/k$ una extensión finita y separable. Entonces $E/k$ tiene finitos cuerpos intermedios.
\begin{proof}
La idea va a ser meter a $E$ en una \textit{clausura normal} y ahí usar la observación anterior (pues obtendríamos una extensión finita y Galois). Para hacer esto, si $E=k(\alpha_1,\ldots , \alpha_r)$ con cada $\alpha_i$ separables, construímos $E'$ agregando a $E$ las raíces de $m(\alpha_i,k)$ (en $\overline{E}$) para cada $1\leq i\leq r$. Esta extensión queda finita pues agregamos finitas raíces. Entonces, tenemos la torre $k\subseteq E\subseteq E'$. Notemos que $E'$ es separable sobre $k$ pues todos los generadores lo son (esto es verdad pues los minimales de los generadores son los $m(\alpha_i,k)$ y esos son separables porque los $\alpha_i$ lo son). Además, es normal porque la definimos para que así lo sea. Entonces, $E'/k$ es Galois y finita, y por la observación anterior tiene finitos cuerpos intermedios. Como $E\subseteq E'$, los cuerpos intermedios de $E/k$ son cuerpos intermedios de $E'/k$, y así finitos. Como queríamos probar.
\end{proof}
\end{prop}

\begin{obs}
Si $k$ es perfecto, entonces toda extensión finita $E/k$ tiene finitos cuerpos intermedios. En efecto, esto es obvio pues si un cuerpo es perfecto debe ser separable.
\end{obs}

\begin{prop}
Sea $k$ un cuerpo.  Si $G\subseteq k^\times$ es un subgrupo de las unidades de $k$ tal que $|G|<+\infty$, entonces $G$ es cíclico.
\begin{proof}
Notemos que $G$ es abeliano y finito. Entonces, por el Teorema de Estructura (para $\ZZ$-módulos) tenemos que $G\simeq G_{p_1}\times\ldots\times G_{p_r}$ donde cada $G_{p_i}$ es una componentes $p_i$-primarias de $G$. Como $p_i\neq p_j$, basta ver que cada $G_{p_i}$ es cíclico para cada $1\leq i\leq r$. Sea $a_i\in G_{p_i}$ tal que $\ord a_i$ es máximo. Luego, $\ord a_i = p_i^{m_i}$ para algún $m_i\in\NN_0$. Notemos que $x^{p_i^{m_i}}-1\in k[x]$ tiene como máximo $p_i^{m_i}$ raíces. Además, para cada $x\in G_{p_i}$ tenemos que $x^{p_i^{m_i}}=1$. Como $|\langle a_i\rangle| = \ord a_i = p_i^{m_i}$ y así $\langle a_i\rangle = G_{p_i}$. Como queríamos probar.
\end{proof}
\end{prop}

\begin{cor}
Si $k$ es un cuerpo finito, entonces $k^\times$ es cíclico.
\begin{proof}
Es aplicar directamente el teorema anterior.
\end{proof}
\end{cor}

\begin{teo}[Teorema del Elemento Primitivo]
Sea $E/k$ una extensión finita. Si $E/k$ tiene finitos cuerpos intermedios, entonces existe $\alpha\in E$ tal que $E=k(\alpha)$. En particular, si $k$ es un cuerpo perfecto (por la observación anterior) luego $E$ es una extensión simple.
\begin{proof}
Si $k$ es finito, entonces $E$ también es un cuerpo finito (pues son finitas combinaciones lineales de finitos elementos de la base). Entonces, $E^\times$ es cíclico y así $E=k(\alpha)$ donde $\alpha$ es un generador de $E^\times$.

Si $k$ es infinito, sean $\alpha,\beta\in E$. Consideremos las subextensiones $k(\alpha+c\beta)$ con $c\in k$. Como hay finitos cuerpos intermedios, debemos tener $c_1,c_2\in k$ tales que $k(\alpha + c_1\beta) = k(\alpha + c_2\beta)$ y llamemos $F$ a esa extensión. Esto quiere decir que $\alpha + c_1\beta, \alpha+c_2\beta\in F$ y así la resta también está. Es decir, $(c_2-c_1)\beta \in F$. Como $c_1\neq c_2$, tenemos que $c_2 - c_1 \in k^\times$ y así $\beta\in F$. Esto implica que $\alpha\in F$. Por lo tanto, probamos que $k(\alpha,\beta)\subseteq F$. Como $F\subseteq k(\alpha,\beta)$ trivialmente,  debemos tener que $k(\alpha,\beta) = F$. Procediendo recursivamente, vemos que $E=k(\alpha_1,\ldots,\alpha_r) = k(\alpha_1 + c_2\alpha_2 + \ldots + c_r\alpha_r)$. El teorema sigue.
\end{proof}
\end{teo}

\begin{obs}
Vale la recíproca: si $\alpha\in\overline{k}$ es algebraico sobre $k$, entonces $k(\alpha)/k$ tiene finitos cuerpos intermedios.
\begin{proof}
Sea $\overline{k}$ una clausura algebraica de $k$ y $\alpha\in\overline{k}$. Sea $F$ un cuerpo intermedio de $k(\alpha)/k$. Consideremos entonces $p_F = m(\alpha,F)$. Tenemos que $p_F\mid p_k$ en $\overline{k}[x]$. Esto implica que existen finitos $p_F$. Veamos que si $p_F = p_{F'}$ entonces $F=F'$. En efecto, si tenemos que $p_F = a_n x^n + \ldots + a_1x+a_0$ defino $F_0 = k(a_0,\ldots,a_n)$. Como $k\subseteq F$ y cada $a_i\in F$, entonces $F_0\subseteq F$. Es fácil ver que $k(\alpha) = F(\alpha)=F_0(\alpha)$. Pero notemos que $$[F(\alpha):F] = \gr p_F = \gr p_{F_0} = [F_0(\alpha):F_0]$$ Esto implica que $[F:k] = [F_0 : k]$ y como $F_0\subseteq F$ debemos tener $F_0=F$. Por lo tanto, si $p_F = p_{F'}$ tendríamos que $F_0 = F_0'$ y así $F = F'$. Como queríamos probar.
\end{proof}
\end{obs}

\begin{cor}
Una extensión es simple si y sólo si tiene finitos cuerpos intermedios.
\begin{proof}
Simplemente es combinar la observación y el teorema anterior.
\end{proof}
\end{cor}

\begin{teo}
Sea $E/k$ una extensión algebraica tal que todo polinomio con coeficientes en $k$ tiene alguna raíz en $E$. Entonces, $E$ es algebraicamente cerrado.
\begin{proof}
Primero, supongamos que $k$ es perfecto. Sea $f\in k[x]$ irreducible y sea $F/k$ su cuerpo de descomposición. Como $k$ es perfecto, por el Teorema del Elemento Primitivo, existe $\alpha\in F$ tal que $F=k(\alpha)$. Por hipótesis, existe $\beta\in E$ raíz de $m(\alpha,k)$. Pero $F/k$ es normal por ser un cuerpo de descomposición, y esto implica que $k(\alpha)=k(\beta)\subseteq E$. Esto es, todo el cuerpo de descomposición de $f$ está metido en $E$. Por lo tanto, todas las raíces de $f$ están en $E$. Es decir, todas las raíces de cualquier polinomio están en $E$ y esto implica que es algebraicamente cerrado.

Ahora, si $k$ no es perfecto, en particular tenemos que $\car k = p>0$. Consideremos el conjunto $P = \{x\in E : \exists\, n\in\NN \text{ tal que } x^{p^n}\in k\}$. Es fácil ver que $P$ es un cuerpo (pues si $x,y\in P$, tomo el máximo de los $n_x,n_y$ y lo llamo $m$ y tengo que $(x+y)^{p^m} = x^{p^m}+y^{p^m}\in k$ y análogamente todas las otras verificaciones). Además, es claro por la definición de $P$ que $k\subseteq P\subseteq E$. Veamos que $P$ es perfecto. Sea $\alpha\in P$ y $n\in\NN$ tal que $\alpha^{p^n}\in k$. Consideremos entonces el polinomio $f(x)=x^{p^{n+1}}-\alpha^{p^n}\in k[x]$. Como todo polinomio $f\in k[x]$ tiene alguna raíz en $E$, existe $y\in E$ tal que $f(y)=0$ y así $y^{p^{n+1}}-\alpha^{p^n}=0$. Pero esto es $(y^p - \alpha)^{p^n} = 0$. Como $y^{p^{n+1}} = \alpha^{p^n}\in k$, tenemos que $y\in P$ y después tenemos que $y^p = \alpha$. Entonces, $P^p = P$ y así $P$ es perfecto. Notemos que esto implica que $E$ también es perfecto.

Veamos que $E/P$ tiene la propiedad de que todo polinomio irreducible tiene alguna raíz en $E$. Sea $f\in P[x]$ un polinomio irreducible con $f(x) = a_n x^n + \ldots + a_1x+a_0$. Como $a_i\in P$ para cada $0\leq i\leq n$, existen $n_i\in\NN$ tales que $a_i^{p^{n_i}}\in P$. Sea $N = \max_{0\leq i\leq n} n_i$. Entonces, si $g(x) = a_n^{p^N} x^n +\ldots + a_1^{p^N}x + a_0^{p^N}\in k[x]$. Por hipótesis, tenemos alguna raíz $\alpha\in E$ de este polinomio. Como $E$ es perfecto, existe $\beta\in E$ tal que $\beta^{p^N}=\alpha$. Es decir, $0=a_n^{p^N} \alpha^n + \ldots + a_1^{p^N} \alpha + a_0^{p^N} = a_n^{p^N}(\beta^n)^{p^N} + \ldots + a_1^{p^N} \beta^{p^N} + a_0^{p^N} = (a_n \beta^n + \ldots + a_1\beta + a_0)^{p^N}$. Por lo tanto, $f(\beta)=0$ y así $f$ tiene una raíz en $E$ como queríamos.

Como $E/P$ cumple la condición del enunciado y $P$ es perfecto, ya vimos que $E$ debe ser algebraicamente cerrado. Como queríamos probar.
\end{proof}
\end{teo}


\begin{lem}
Sea $E/k$ separable. Si existe $n\in\NN_0$ tal que $[k(\alpha):k]\leq n$ para todo $\alpha\in E$, entonces $[E:k]\leq n$. En particular, la extensión es finita.
\begin{proof}
Sea $\gamma\in E$ tal que $k(\gamma)/k$ tiene grado máximo. Si $E\neq k(\gamma)$, entonces existe $\beta\in E$ tal que $\beta\notin k(\gamma)$. Consideremos entonces $k(\gamma,\beta)$. Esto es una extensión finita y separable, y así por el Teorema del Elemento Primitivo, existe $\omega$ tal que $k(\gamma,\beta) = k(\omega)$. Pero sabemos que $[k(\omega):k] = [k(\gamma,\beta):k] > [k(\gamma):k]$, que tenía grado máximo. Absurdo, que vino de suponer que $E\neq k(\gamma)$.
\end{proof}
\end{lem}

\begin{teo}[Artin]
Sea $E$ un cuerpo y $G\subseteq\Aut E$ un subgrupo finito. Si $k=E^G$, entonces $E/k$ es de Galois y $\Gal(E/k) = G$, y así $[E:k]=|G|$.
\begin{proof}
Sea $\alpha\in E$. Sea $\{\sigma_1,\ldots , \sigma_r\}\subseteq G$ maximal respecto de la propiedad que $\sigma_i(\alpha)\neq\sigma_j(\alpha)$ para todo $i\neq j$. Si $\tau\in G$, entonces tenemos $$\{\tau\circ\sigma_1(\alpha),\ldots , \tau\circ\sigma_r(\alpha)\} = \{\sigma_1(\alpha),\ldots ,\sigma_r(\alpha)\}$$ pues si $\tau\circ\sigma_i(\alpha) \neq \sigma_j(\alpha)$ para todo $1\leq j\leq r$, entonces $\{\sigma_1,\ldots , \sigma_r, \tau\circ\sigma_i\}$ cumple la propiedad, y esto contradice la maximalidad. Además, como $\tau$ es una biyección queda que todos los $\tau\circ\sigma_i(\alpha)$ son distintos.
Notemos también que $\alpha\in \{\sigma_1(\alpha),\ldots , \sigma_r(\alpha)\}$, ya que simplemente tomando $\tau = \sigma_1^{-1}$ obtenemos $\sigma^{-1}\circ \sigma(\alpha) = \alpha$.

Ahora bien, si $f_\alpha = \displaystyle\prod_{i=1}^r (x-\sigma_i(\alpha))\in E[x]$, tenemos que $\alpha$ es raíz de $f_\alpha$. Es claro también que es separable (pues cada $\sigma_i(\alpha)\neq \sigma_j(\alpha)$) y además se está factorizando linealmente en $E[x]$. Pero más aún, $f_\alpha\in k[x]$. En efecto, si $\tau\in G$, tenemos que $f_\alpha^\tau = f_\alpha$ pues $\{\tau\circ\sigma_i(\alpha)\}_{1\leq i \leq r}$ simplemente es una permutación de $\{\sigma_i(\alpha)\}_{1\leq i \leq r}$. Como al aplicar $\tau$ simplemente permuté las raíces, los coeficientes deben quedar fijos y así, como son dejados fijos por todos los elementos de $G$, esto implica $f_\alpha\in k[x]$.

Como el $\alpha\in E$ era arbitrario, esto implica que la extensión $E/k$ es de Galois. Para ver la otra afirmación, notemos que siempre tenemos la inclusión $G\subseteq \Gal(E/k) = \Gal(E/E^G)$ pues simplemente es un juego de palabras ($G$ está entre los automorfismos que dejan fijos a las cosas dejadas fijas por $G$). Para ver la igualdad, como para todo $\alpha\in E$ tenemos que $[k(\alpha):k]\leq n$ pues $m(\alpha,k)\mid f_\alpha$ y $\gr f_\alpha = r\leq |G|$, el lema anterior implica que $|\Gal(E/k)| = [E:k]\leq |G|$. Es decir, $G$ está incluído en un conjunto de su mismo cardinal. Entonces $\Gal(E/k)=G$, y el teorema sigue.
\end{proof} 
\end{teo}

\begin{cor}[Teorema Fundamental de la Teoría de Galois II]
Sea $E/k$ una extensión de Galois y finita. Entonces, la función $$ \varphi:\{\text{Cuerpos intermedios de }E/k\}\to \{\text{Subgrupos de }\Gal(E/k)\}$$ dada por $\varphi(F)=\Gal(E/F)$ es biyectiva con inversa $\varphi^{-1}(H) = E^H$.
\begin{proof}
La inyectividad ya la vimos (bajo el nombre de Teorema Fundamental de la Teoría de Galois I). Ahora, notemos que por el teorema anterior, $E/E^H$ es Galois y $\Gal(E/E^H) = H$, lo que implica que $\varphi^{-1}(H)=E^H$.
\end{proof}
\end{cor}

\begin{obs}
Sea $F/k$ una extensión de Galois y $\sigma:F\to F'$ un isomorfismo de cuerpos. Entonces, $F/k\stackrel{\sigma}{\longrightarrow} F'/\sigma(k)$ es un isomorfismo. Consideremos ahora la conjugación $\widehat{\sigma}:\Gal(F/k)\to \Gal(F'/\sigma(k))$ dada por $\widehat{\sigma}(\varphi) = \sigma\varphi\sigma^{-1}$. Luego, es fácil ver que $\Gal(F'/\sigma(k)) = \widehat{\sigma}(\Gal(F/k))$.
\end{obs}

\begin{prop}
Sea $E/k$ Galois y sea $F/k$ una subextensión. Entonces:
\begin{itemize}
\item $F/k$ es normal si y sólo si $\Gal(E/F)\triangleleft \Gal(E/k)$.
\item Si $F/k$ es normal, entonces $\Gal(F/k) \simeq \Gal(E/k)/\Gal(E/F)$.
\end{itemize}
\begin{proof}

Primero, supongamos que $F/k$ es normal. Luego, $\rho:\Gal(E/k)\to \Gal(F/k)$, $\sigma\mapsto\left.\sigma\right|_{F}$ resulta un morfismo de grupos. En efecto, esto está bien definido por la normalidad de $F/k$. Además, $\rho$ resulta un epimorfismo pues todo $\sigma:F/k\to \overline{k}/k$ se extiende a un morfismo $\overline{\sigma}:E/k\to\overline{k}/k$ ya que $E/F$ es algebraica. Como $E/F$ es normal, resulta que $\overline{\sigma}\in \Gal(F/k)$. Entonces, $\rho(\overline{\sigma})=\sigma$ y esto prueba que es epimorfismo. Ahora bien, como $\ker \rho = \{\omega\in\Gal(E/k) : \left.\omega\right|_{F}=\id\} = \Gal(E/F)$ se sigue que $\Gal(E/F)\triangleleft \Gal(E/k)$ por ser el núcleo de un morfismo que sale de ahí. Finalmente, $$\Gal(F/k) = \im \rho \simeq \Gal(E/k)/\ker\rho = \Gal(E/k)/\Gal(E/F)$$ en virtud del Primer Teorema de Isomorfismo.

Resta ver la vuelta. Es decir, si $\Gal(E/F)\triangleleft\Gal(E/k)$, entonces $F/k$ es normal. Sea $\sigma:F/k\to \overline{k}/k$ un morfismo de extensiones. Debo ver que $\sigma(F)=F$. Extendamos $\sigma$ a $\overline{\sigma}:E/k\to\overline{k}/k$. Como $E/k$ es normal, se tiene que $\overline{\sigma}\in\Gal(E/k)$. Esto implica que $\overline{\sigma}$ es un isomorfismo de cuerpos, y esto a su vez implica que $E/F\stackrel{\overline{\sigma}}{\longrightarrow} E/\sigma(F)$ es un isomorfismo y así $\Gal(E/\sigma(F)) = \widehat{\overline{\sigma}}(\Gal(E/F))$ por la observación anterior. Pero como este subgrupo es normal, al conjugar obtengo lo mismo. Es decir, $\Gal(E/\sigma(F)) = \Gal(E/F)$. Por el Teorema Fundamental de la Teoría de Galois I se debe tener que $F=\sigma(F)$. Y estamos.

\end{proof}
\end{prop}

\begin{ex}
Calcular el grupo de Galois de $\QQ(\sqrt[3]{2},\xi_3)/\QQ$ donde $\xi_3$ es una raíz cúbica de la unidad.

Esta extensión es de Galois pues es algebraica (sus elementos son algebraicos), es separable (pues $\car \QQ=0$) y es normal pues es el cuerpo de descomposición de $x^3 - 2$. Ahora, es fácil ver que $[\QQ(\sqrt[3]{2},\xi_3):\QQ]=6$ (¡ejercicio!). Esto quiere decir que el grupo de Galois $\Gal(\QQ(\sqrt[3]{2},\xi_3)/\QQ)$ tiene $6$ elementos y así sabemos que debe ser $S_3$ o $\ZZ_6$. Pero si fuera $\ZZ_6$, sería abeliano y así sabemos que todos sus subgrupos son normales y así todas las subextensiones lo serían. Pero $\QQ(\sqrt[3]{2})/\QQ$ no es normal. Entonces, el grupo de Galois debe ser $S_3$.
\end{ex}

\begin{defn}
Una extensión $E/k$ finita se dice \textbf{abeliana} (resp. cíclica) si es Galois y su grupo de Galois $\Gal(E/k)$ es abeliano (resp. cíclico).
\end{defn}

\begin{obs}
Si $E/k$ es abeliana (resp. cíclica) entonces toda subextensión $F/k$ es abeliana (resp. cíclica).
\begin{proof}
En efecto, toda extensión $F/k$ es normal pues todo subgrupo de un grupo abeliano (si es cíclico es, en particular, abeliano) es normal. Entonces, toda subextensión es Galois y como $\Gal(F/k)$ es un cociente de $\Gal(E/k)$, debe ser abeliano (resp. cíclico).
\end{proof}
\end{obs}

\begin{ex}
Calculemos $\Gal(\QQ(\sqrt{2},\xi_3)/\QQ)$ y todos los cuerpos intermedios de esa subextensión. En efecto, $m(\sqrt{2},\QQ)=x^2 - 2$ y $m(\xi_3,\QQ)=x^2 + x + 1$. Como $\car \QQ=0$, la extensión es separable y es normal pues es el cuerpo de descomposición de esos dos minimales (pues ambos tienen grado 2). Es fácil ver que $[\QQ(\sqrt{2},\xi_3):\QQ]=4$ (en efecto, $\xi_3\notin \QQ(\sqrt{2})$ por ser complejo). Esto quiere decir que $\Gal(\QQ(\sqrt{2},\xi_3)/\QQ)$ es un grupo de orden $4$. Por lo tanto, o bien es $\ZZ_4$ o $\ZZ_2\oplus\ZZ_2$. Pero si el grupo de Galois es $\{\sigma_1,\sigma_2,\sigma_3,\sigma_4\}$ como los morfismos quedan determinados por su acción en los generadores tenemos \begin{align*} \sigma_1(\sqrt{2})=\sqrt{2} &\text{ y } \sigma_1(\xi_3)=\xi_3 \\ \sigma_2(\sqrt{2})=-\sqrt{2} &\text{ y } \sigma_2(\xi_3)=\xi_3 \\ \sigma_3(\sqrt{2})=\sqrt{2} &\text{ y } \sigma_3(\xi_3)=\xi_3^2 \\ \sigma_4(\sqrt{2})=-\sqrt{2} &\text{ y } \sigma_4(\xi_3)=\xi_3^2 \end{align*} Es fácil ver que $\sigma_i^2 =\id$ para cada $i$, es decir, no hay ningún elemento de orden 4, lo que implica que el grupo de Galois era $\ZZ_2\oplus\ZZ_2$. Para calcular los cuerpos intermedios, simplemente debemos ver los elementos dejados fijos por cada subgrupo del grupo de Galois, y en este caso esto es por cada uno de esos morfismos. Es fácil ver qué es lo dejado fijo por $\sigma_1=\id$ es todo $\QQ(\sqrt{2},\xi_3)$, que lo dejado fijo por todo el grupo de Galois es $\QQ$, que lo dejado fijo por $\{\id,\sigma_2\}$ es $\QQ(\xi_3)$ y lo dejado fijo por $\{\id,\sigma_3\}$ es $\QQ(\sqrt{2})$. Sólo falta ver lo dejado fijo por $\{\id,\sigma_4\}$, es decir, lo dejado fijo por $\sigma_4$.

Sea $\alpha\in\QQ(\sqrt{2},\xi_3) = (\QQ(\xi_3))(\sqrt{2})$. Si queda fijo por $\sigma_4$ entonces $$\alpha = (a+b\xi_3) + (c+d\xi_3)\sqrt{2} = \sigma_4(\alpha) = (a + b\xi_3^2) - (c+d\xi_3^2)\sqrt{2}$$ Por la unicidad del desarrollo en base, como $\{1,\sqrt{2}\}$ es base de $\QQ(\xi_3)(\sqrt{2})$ como $\QQ(\xi_3)$-espacio vectorial, tenemos que $a+b\xi_3^2 = a+b\xi_3$ y $-c-d\xi_3^2 = c+d\xi_3$. Como $\xi_3^2 + \xi_3 + 1=0$, podemos reemplazar eso y ver que $2c=d$, y de la primera ecuación que $a=b=0$.  Entonces, los $\alpha$ que quedan fijos son los de la forma $c(1+2\xi_3)\sqrt{2}$. Entonces, $\QQ((1+2\xi_3)\sqrt{2})/\QQ$ es la extensión algebraica que nos faltaba.
\end{ex}

\begin{ex}
Si $\xi_p$ es una raíz $p$-ésima de la unidad, con $p$ primo, calculemos entonces $\Gal(\QQ(\xi_p)/\QQ)$. Notemos que $f(x)=x^{p-1} + \ldots + x+1 = \dfrac{x^p - 1}{x-1}$ anula a $\xi_p$. Si vemos que es irreducible, deberá ser el minimal. Notemos que $f(x)$ es irreducible si y sólo si $f(x+1)$ lo es. Pero $f(x+1) = \dfrac{(x+1)^p - 1}{x} = x^p + \binom{p}{1}x^{p-1}+\ldots + \binom{p}{2}x + p$. Por el criterio de Eisenstein es irreducible y así $f(x)$ lo es. Entonces, $[\QQ(\xi_p):\QQ]=p-1$, y así $\Gal(\QQ(\xi_p)/\QQ)$ tiene orden $p-1$. Por otra parte, si $\sigma\in\Gal(\QQ(\xi_p)/\QQ)$, queda determinado por su valor $\sigma(\xi_p)$. Como esto tiene que dar otra raíz del polinomio minimal de $\xi_p$, debemos tener que $\sigma(\xi_p)=\xi_p^i$ para algún $1\leq i \leq p-1$. Esto determina un monomorfismo $\Gal(\QQ(\xi_p)/\QQ)\to (\FF_p)^\times$ por $\sigma\mapsto i$ si $\sigma(\xi_p)=\xi_p^i$. Como estos grupos finitos tienen el mismo orden, ese monomorfismo en realidad es un isomorfismo y así el grupo de Galois es isomorfo a $\FF_p^\times \simeq \ZZ_{p-1}$.
\end{ex}

\begin{defn}
Sea $k$ un cuerpo y $f\in k[x]$ un polinomio no constante y separable. Sea $E/k$ el cuerpo de descomposición de $f$ sobre $k$. Como $E/k$ es Galois y finita, podemos considerar el \textbf{grupo de Galois de} $f$ sobre $k$ como $G_f = \Gal(E/k)$.
\end{defn}

\begin{obs}
Si $\gr f=n$, dado $\sigma\in G_f$, tenemos que $\sigma$ queda determinada por su valor en las raíces y determina una permutación de las raíces de $f$. Se tiene entonces un monomorfismo $G_f\hookrightarrow S_n$, $\sigma\mapsto \{\text{permutación de las raíces correspondientes}\}$. Esto es, $G_f$ es isomorfo a un subgrupo de $S_n$ y así $|G_f|\mid |S_n| = n!$ por Lagrange. Esto es, $[E:k]\mid n!$.
\end{obs}

\begin{ex}
Calculemos $G_f$ para $f=x^5 - 2\in\QQ[x]$. Notemos que $E=\QQ(\sqrt[5]{2},\xi_5)$ es el cuerpo de descomposición de $f$ sobre $\QQ$. Tenemos el siguiente diamante: 
\begin{center}
\begin{tikzcd}[row sep=1.3em,column sep=1.3em,minimum width=2em]
 & \QQ(\sqrt[5]{2},\xi_5)\arrow[dash]{rd} &  \\
\QQ(\sqrt[5]{2}) \arrow[dash]{ur}\arrow[dash]{dr} & & \QQ(\xi_5) \\
& \QQ\arrow[dash]{ur} &
\end{tikzcd}
\end{center}
Entonces, $[E:\QQ]=5\cdot 4 = 20$ y así $|G_f|=20$. Como $\QQ(\xi_5)/\QQ$ es normal, tenemos que $T=\Gal(\QQ(\sqrt[5]{2},\xi_5)/\QQ(\xi_5))$ es un subgrupo normal del grupo de Galois y tiene orden $5$. Además, $H=\Gal(\QQ(\sqrt[5]{2},\xi_5)/\QQ(\sqrt[5]{2})$ tiene orden $4$ y así es $\ZZ_4$ o $\ZZ_2\oplus\ZZ_2$. Como $T\cap H = \{1\}$, tenemos que el grupo de Galois es el producto semidirecto $T\rtimes H$. Esto es $H\simeq \Gal(\QQ(\sqrt[5]{2},\xi_5)/\QQ)/T$, que el lado derecho es precisamente el grupo de Galois $\Gal(\QQ(\xi_5)/\QQ)$. Esto implica que $H$ es cíclico y así $H\simeq \ZZ_4$. Es fácil ver entonces que la presentación del grupo $\Gal(\QQ(\sqrt[5]{2},\xi_5)/\QQ)$ es $\langle \sigma,\tau | \sigma^5, \tau^4, \tau\sigma\tau^{-1}\sigma^{-2} \rangle$.
\end{ex}

\begin{prop}
Sea $f\in\QQ[x]$ irreducible de grado $p$ primo. Si $f$ tiene exactamente dos raíces no reales en $\CC$, entonces $G_f\simeq S_p$.
\begin{proof}
Sabemos que $S_n = \langle (1,2,\ldots , n),(1,2)\rangle$. Es decir, está generado por un $n$-ciclo y una trasposición en particular. Notemos que si $p$ es primo, $S_p$ está generado por cualquier $p$-ciclo y cualquier trasposición. 

Como $f\in\QQ[x]$, si $\beta\in\CC$ es raíz de $f$ y $\beta\notin\RR$ entonces $\overline{\beta}$ también es raíz. Las raíces de $f$ son $\{\alpha_1,\ldots ,\alpha_{p-2}\}$ las reales y $\{\beta,\overline{\beta}\}$ las complejas. Sea $E\subseteq\CC$ el cuerpo de descomposición de $f$ y sea $\tau:E\to E$, $\tau(x)=\overline{x}$. Es claro que $\tau\in G_f$ es una trasposición (pues sólo cambia $\beta$ con $\overline{\beta}$ y las otras quedan fijas). Como $[\QQ(\alpha_i):\QQ]=p$, tenemos que $p\mid |G_f|$. Por Cauchy, existe algún elemento $\sigma\in G_f$ de orden $p$. Entonces, $\langle \sigma\rangle$ tiene orden $p$. Me conseguí un $p$-ciclo y una trasposición en $G_f$, y así puedo generar todo $S_p$. Y listo.
\end{proof}
\end{prop}

\begin{teo}[Teorema Fundamental del Álgebra]
$\CC$ es algebraicamente cerrado.
\begin{proof}

Sabemos que si $\alpha\in\RR$ y $\alpha\geq 0$, entonces existe $\beta\in\RR$ tal que $\beta^2 =\alpha$ y que todo polinomio $f\in\RR[x]$ de grado impar tiene alguna raíz en $\RR$ (ambos son consecuencia del Teorema de los Valores Intermedios). Veamos que todo $\alpha\in\CC$ tiene alguna raíz cuadrada en $\CC$. En efecto, si $\alpha = a+bi$, tomemos $c\in\RR$ tal que $c^2 = \dfrac{a+\sqrt{a^2+b^2}}{2}$ y $d\in\RR$ tal que $d^2 = \dfrac{-a+\sqrt{a^2+b^2}}{2}$ (hay que chequear que todas las cosas a las que le vemos la raíz sea positiva, y que siempre que miramos raíz miramos la raíz positiva). Es fácil chequear que $c^2 - d^2 = a$ y que $(2cd)^2 = b^2$. Esto implica que $2cd=\pm b$. Elegimos los signos de $c,d$ de manera que $2cd = b$. Entonces, $(c+di)^2 = a + bi$.

Algebraicamente, este hecho se traduce como que $\CC$ no tiene extensiones cuadráticas. En efecto, si $k/\CC$ fuera una extensión cuadrática (como $\car \CC = 0\neq 2$), $k=\CC(\alpha)$ con $\alpha^2\in\CC$. Pero entonces $\alpha\in\CC$ y así $k=\CC$. Absurdo.

Por otra parte, que todo polinomio de grado impar con coeficientes reales tenga alguna raíz real implica que no existen extensiones de grado impar de $\RR$. En efecto, sea $k/\RR$ es una extensión de grado impar. Como $\RR$ es perfecto, por el Teorema del Elemento Primitivo, $k=\RR(\alpha)$ para algún $\alpha$. Pero $m(\alpha,\RR)$ tiene grado $[k:\RR]$, que es impar y así tiene una raíz en $\RR$. Absurdo, pues no sería irreducible.

Para ver que $\CC$ es algebraicamente cerrado, veamos que no tiene extensiones finitas propias. En efecto, sea $F/\CC$ una extensión finita. Como $\CC/\RR$ es finita y $[\CC:\RR]=2$, entonces $F/\RR$ es una extensión finita. Como $\car\RR=0$, tenemos que $F/\RR$ es separable. Sea $E/\RR$ la clausura normal de $F/\RR$. Esto implica que $E/\RR$ es Galois.
Además, sabemos que $|\Gal(E/\RR)| = [E:\RR] = 2^h n$ con $h\geq 1$ y $n$ impar. Por el Teorema de Sylow, existe $H\subseteq \Gal(E/\RR)$ con $|H|=2^h$. Por el Teorema Fundamental de la Teoría de Galois, tenemos que $[E^H:\RR]=n$. Pero como $n$ es impar, ya vimos que esto implica que $E^H = \RR$. Por lo tanto, $[E:\RR]=2^h$. Como $[\CC:\RR]=2$, tenemos que $[E:\CC]=2^{h-1}$. Si $h>1$,  como $E/\CC$ es Galois, tenemos que $|\Gal(E/\CC)| = 2^{h-1}$. Luego, (por un lema de $p$-grupos) existe un subgrupo $T\triangleleft \Gal(E/\CC)$ tal que $|T| = 2^{h-2}$. Pero entonces, $[E^T : \CC]=2$. Absurdo, porque no hay extensiones cuadráticas propias de $\CC$. Es decir, $h=1$. Pero esto es $[E:\CC] = 1$ y así no hay extensiones finitas propias de $\CC$. Como queríamos probar.

\end{proof}
\end{teo}

\section{Teoría de Galois Infinita}

En la sección anterior enunciamos el Teorema Fundamental de la Teoría de Galois. En la primera parte no necesitamos en ningún momento la finitud de la extensión, pero para la segunda parte sí. ¿Se podrá probar el teorema sin esa condición? Veamos un contraejemplo:

\begin{ex}
Sea $S=\{\sqrt{p} : \,p\text{ primo}\}$. Consideremos la extensión $\QQ(S)/\QQ$. Es claro que $\QQ(S)/\QQ$ es infinita y es Galois (es separable por ser $\QQ$ perfecto y es normal pues la extensión generada por cada $s\in S$ es normal por ser cuadrática). Ahora bien, notemos que $\Gal(E/k)\simeq \displaystyle\prod_{p\text{ primo}} \ZZ_2$ pues queda determinado por dónde mando a cada $\sqrt{p}$ y cada uno puede ir a $\pm\sqrt{p}$. Entonces, $V=\Gal(\QQ(S)/\QQ)$ es un $\ZZ_2$-espacio vectorial con base no numerable. Luego, el espacio dual $V^*=\Gal(\QQ(S)/\QQ)^*=\{\varphi:\Gal(\QQ(S)/\QQ)\to\ZZ_2 \text{ lineales}\}$ es no numerable. Pero notemos además que cada $\varphi\in V^*$ queda unívocamente determinado por $\ker\varphi$ (pues si sabemos en qué valores va al $0$, en el resto debe ir al $1$). Ahora bien, si $\varphi\neq 0$, por el Teorema de Isomorfismo tenemos que $V/\ker\varphi \simeq \ZZ_2$. Esto quiere decir que $\Gal(\QQ(S)/\QQ)$ tiene no numerables subgrupos de índice $2$, es decir, los $\ker\varphi$ para $\varphi\in V^*$. Si valiera la correspondencia, tendríamos no numerables extensiones cuadráticas de $\QQ$. Pero estas son de la forma $\QQ(\sqrt{\alpha})$ con $\alpha^2\in\QQ$. Esto nos dice que son numerables. Absurdo.
\end{ex}

\begin{defn}
Un \textbf{conjunto dirigido} $(I,\leq)$ es un conjunto parcialmente ordenado tal que $\forall i,j\in I$ existe $k\in I$ tal que $i,j\leq k$.
\end{defn}

\begin{defn}
Sea $\mathscr{C}$ una categoría. Un \textbf{sistema inverso} en $\mathscr{C}$ indexado por un conjunto dirigido $(I,\leq)$ consiste de una familia de objetos $\{A_i\}_{i\in I}$ y una familia de morfismos $\{\varphi_{ij}\}_{i\leq j}$, $\varphi_{ij}:A_j\to A_i$ tales que: \begin{itemize}\item $\varphi_{ii} = 1_{A_i}$ para todo $i\in I$\item $\varphi_{ik} = \varphi_{ij}\circ\varphi_{jk}$ para todos $i\leq j\leq k$. Es decir, el siguiente diagrama conmuta:
\begin{center}
\begin{tikzcd}[row sep=1.3em,column sep=1.3em,minimum width=2em]
 & A_k \arrow[]{rd}[font=\normalsize]{\varphi_{ik}}\arrow[]{ld}[left, font=\normalsize]{\varphi_{jk}} &  \\
A_j \arrow[]{rr}[font=\normalsize]{\varphi_{ji}} & & A_i
\end{tikzcd}
\end{center}\end{itemize}
Notaremos por $\{A_i,\varphi_{ij}\}_{I}$ al sistema inverso con objetos $\{A_i\}_{i\in I}$ y morfismos $\{\varphi_{ij}\}_{i\leq j}$.
\end{defn}

\begin{defn}
Sea $\{A_i,\varphi_{ij}\}_{I}$ un sistema inverso. El \textbf{límite inverso} (proyectivo) del sistema es un objeto $A=\varprojlim A_i$ junto con morfismos $p_i:A\to A_i$ tal que $p_i = \varphi_{ij}\circ p_j$ para todo $i\leq j$ y que es minimal respecto de esta propiedad. Es decir, para todo objeto $B$ y morfismos $q_i:B\to A_i$ tal que si $q_i=\varphi_{ij}\circ q_j$ para todo $i\leq j$, entonces existe un único morfismo $q:B\to A$ tal que el siguiente diagrama conmuta:
\begin{center}
\begin{tikzcd}[row sep=1.3em,column sep=1.3em,minimum width=2em]
 & B\arrow[dashed]{d}[font=\normalsize]{\exists ! q}\arrow[bend right]{ddl}[left, font=\normalsize]{q_j}\arrow[bend left]{ddr}[right, font=\normalsize]{p_i} & \\
 & A \arrow[]{rd}[right, font=\normalsize]{p_i}\arrow[]{ld}[left, font=\normalsize]{p_j} &  \\
A_j \arrow[]{rr}[font=\normalsize]{\varphi_{ji}} & & A_i
\end{tikzcd}
\end{center}
\end{defn}

\begin{obs}
Por el argumento clásico de las propiedades universales, si este objeto existe, es único. En efecto, cambiamos $A$ y $B$ de lugar y componemos los morfismos que obtenemos entre $A$ y $B$ para obtener morfismos de $A$ en $A$ y $B$ en $B$ que hacen conmutar el diagrama. Como $1:A\to A$ y $1:B\to B$ hacen conmutar al diagrama, deben ser que esas composiciones daban la identidad.

Ahora bien, para la existencia, en las categorías $\mathrm{Sets}, \mathrm{Group}$ y $\mathrm{Top}$, simplemente debemos considerar $$\varprojlim A_i = \left\{(x_i)_{i\in I}\in\displaystyle\prod_{i\in I}A_i : \varphi_{ij}(x_j)=x_i\;\forall i\leq j\right\}$$ que tautológicamente cumplirá. En categorías arbitrarias, podría no existir el límite inverso.
\end{obs}

\begin{prop}
Si $\{X_i,\varphi_{ij}\}_I$ es un sistema inverso de espacios topológicos $\mathrm{T}_2$, entonces $\varprojlim X_i\subseteq\displaystyle\prod_{i\in I} X_i$ es un subespacio cerrado.
\begin{proof}
Veamos que $(\varprojlim X_i)^c$ es abierto en $\displaystyle\prod_{i\in I} X_i$. Si $(x_i)_{i\in I}\in (\varprojlim X_i)^c$, entonces existen $i\leq j$ tales que $\varphi_{ij}(x_j)\neq x_i$. Como $X_i$ es $\mathrm{T}_2$, tengo abiertos disjuntos $U,V\subseteq X_i$ tales que $x_i\in U$ y $\varphi_{ij}(x_j)\in V$. Consideremos $W$ el abierto del producto que en el $i$-ésimo jugar tiene a $U$, en el $j$-ésimo tiene a $\varphi_{ij}^{-1}(V)$ y en el $k\neq i,j$ tiene a $X_k$. Es trivialmente un abierto de la topología producto. Por contrucción, tenemos que $(x_i)_{i\in I}\in W$. Pero además, es claro que $W\subseteq (\varprojlim X_i)^c$ pues $U,V$ son disjuntos. La proposición sigue.
\end{proof}
\end{prop}

\begin{defn}
Un grupo $G$ se dice \textbf{profinito} si $G\simeq\varprojlim G_i$ donde $\{G_i,\varphi_{ij}\}_I$ es un sistema inverso de grupos finitos.
\end{defn}

\begin{defn}
Un \textbf{grupo topológico} es un grupo $G$ que tiene una topología y tal que $\mu:G\times G\to G$, $\mu(a,b)=ab$ y $\nu:G\to G$, $\nu(a)=a^{-1}$ son continuas.
\end{defn}

\begin{obs}
Si $G$ es un grupo topológico y $a\in G$, entonces $\ell_a:G\to G$, $\ell_a(g)=ag$ y $r_a:G\to G$, $r_a(g)=ga$ son homeomorfismos. En efecto, son trivialmente biyectivas y continuas y sus inversas son $\ell_{a^{-1}}$ y $r_{a^{-1}}$ respectivamente.

En consecuencia, si $H\subseteq G$ es un subgrupo abierto de $G$, para cualquier $a\in G$, las coclases $aH$ resultan abiertas. Además, $H$ resulta cerrado, pues $G$ es la unión disjunta de las coclases, $G=\displaystyle\bigsqcup aH = H \sqcup \bigsqcup_{aH\neq H} aH$ y así $H$ es el complemento de una unión de abiertos.

Si $H\subseteq G$ es un subgrupo cerrado de $G$, podemos concluir que las coclases $aH$ son cerradas. Si además el índice $(G:H)<+\infty$, entonces tendremos que $G=H\sqcup \displaystyle\bigsqcup_{aH\neq H}aH$ y así $H$ es el complemento de una unión finita de cerrados, y así abierto.

Si $\mathscr{B}$ es una base de entornos de $1\in G$ (es decir, para todo $U\subseteq G$ tal que $U$ es abierto y $1\in U$, existe $V\in\mathscr{B}$ tal que $1\in V\subseteq U$) entonces, para todo $a\in G$ tenemos que $a\mathscr{B} = \{aV : V\in\mathscr{B}\}$ es una base de entornos de $a$.
\end{obs}

\begin{obs}
Sea $G=\varprojlim G_i$ un grupo profinito. Si además vemos a $\{G_i\}_{i\in I}$ como espacios topológicos discretos, en particular las $\varphi_{ij}:G_j\to G_i$ resultan continuas (¡porque toda función resulta continua!). Entonces, $G$ como $\varprojlim G_i$ en el contexto de espacios topológicos nos da un espacio topológico. Pero esta topología se comporta bien con la estructura de grupo subyacente de $G$. Es decir, un grupo profinito es un grupo topológico. Esto es pues la multiplicación y la inversión es componente a componente y por la propiedad universal de la topología producto, como cada componente resulta continua, toda la función es continua.
\end{obs}

\begin{prop}
Sea $G$ un grupo profinito. Entonces, $G$ es $\mathrm{T}_2$, compacto y totalmente disconexo.
\begin{proof}
Para ver que es $\mathrm{T}_2$, notemos que $G\subseteq\displaystyle\prod_{i\in I} G_i$. Como $G_i$ es finito y discreto, es $\mathrm{T}_2$, y como productos y subespacios de espacios $\mathrm{T}_2$ son $\mathrm{T}_2$, debemos tener que $G$ es $\mathrm{T}_2$.

Para ver la compacidad, como cada $G_i$ es compacto (es finito), por Tychonoff, $\displaystyle\prod_{i\in I}G_i$ es compacto. Por una proposición anterior $G$ es un subespacio cerrado del producto. Entonces, $G$ es cerrado en un compacto y así compacto.

Ver que es totalmente disconexo es análogo: producto de espacios totalmente disconexos es totalmente disconexo y subespacio también. La proposición sigue.
\end{proof}
\end{prop}

\begin{obs}
En realidad, vale la recíproca. Es decir, si $G$ es un grupo topológico $\mathrm{T}_2$, compacto y totalmente disconexo, entonces $G$ es profinito.
\begin{proof}[Idea de la demostración]
Como $G$ es localmente compacto y totalmente disconexo, se puede probar que $\mathscr{B}=\{H\subseteq G : H\text{ subgrupo abierto de }G\}$ forman una base de entornos de $1$. Además, como es localmente compacto, si $H\subseteq G$ es un subgrupo abierto, el índice $(G:H)<+\infty$. Por lo tanto, $N_H = \displaystyle\bigcap_{g\in G} gHg^{-1}$ es un subgrupo normal abierto de $G$. Luego, $\{N_H : H\in\mathscr{B}\}$ resulta una base de entornos abiertos y normales de $1$. Finalmente, se tiene que $G\simeq \varprojlim G/N_H$, lo que concluiría la demostración.
\end{proof}
\end{obs}

\begin{obs}
Volvamos a la Teoría de Galois. Sea $E/k$ una extensión Galois infinita (en realidad, la misma construcción funciona para extensiones finitas, pero no tiene mucha gracia). Sea $I=\{F : k\subseteq F\subseteq E, \; F/k\text{ Galois finita}\}$, ordenado por $F_1\leq F_2$ si $F_1\subseteq F_2$. Notemos que queda dirigido pues $F_1,F_2\leq F_1F_2$, que sigue siendo Galois y finita.

Sean $a_1,\ldots , a_r\in E$. Luego, existe $F\in I$ tal que $a_i\in F$ para todo $1\leq i\leq r$. Esto simplmente es agregarle a $k$ todos los $a_i$ y las raíces de sus minimales.

Consideremos el siguiente sistema inverso de grupos finitos: $\{\Gal(F/k),\varphi_{F_1F_2}\}_{F\in I}$ con $\varphi_{F_1F_2}:\Gal(F_2/k)\to \Gal(F_1/k)$ dado por la restricción $\varphi_{F_1F_2}(\sigma) = \left.\sigma\right|_{F_1}$ si $F_1\leq F_2$. Esto está bien definido por la normalidad de la extensión. Es fácil ver que esto resulta un sistema inverso.
\end{obs}

\begin{teo}
Las restricciones $\rho_F:\Gal(E/k)\to \Gal(F/k)$ dadas por $\rho_F(\sigma) = \left.\sigma\right|_{F}$, con $F\in I$, inducen un isomorfismo $\rho:\Gal(E/k)\to\varprojlim_{F\in I} \Gal(F/k)$ que está dado por $\rho(\sigma) = (\rho_F(\sigma))_{F\in I}$.
\begin{proof}
Para ver que $\rho$ es monomorfismo, sea $\sigma\in\Gal(E/k)$ tal que $\rho(\sigma)=1$. Es decir, $\left.\sigma\right|_{F}=1$ para todo $F\in I$. Pero dado $x\in E$, existen algún $F\in I$ con $x\in F$. Entonces, $\sigma(x) = \left.\sigma\right|_F(x) = x$ y así $\sigma=1$.

Para ver que $\rho$ es epimorfismo, dado $(\tau_F)_{F\in I}\in\varprojlim_{F\in I}\Gal(F/k)$, podemos considerar $\tau\in\Gal(E/k)$ definido por $\tau(x)=\tau_F(x)$ si $x\in F$. Esto está bien definido pues para cualquier $x\in E$, existe $F\in I$ tal que $x\in F$ y además si $x\in F_1,F_2$, como $F_1,F_2\in I$, tenemos que $\left.\tau_{F_1}\right|_{F_1\cap F_2} = \left.\tau_{F_2}\right|_{F_1\cap F_2}$. Esto implica que $\tau_{F_1}(x)=\tau_{F_2}(x)$ y listo.
\end{proof}
\end{teo}

\begin{obs}
Si $E/k$ es Galois finita, como $\Gal(E/k)$ es el máximo en el conjunto dirigido, salen flechas hacia todas las otras subextensiones Galoisianas (trivialmente finitas) y así cumple la Propiedad Universal del límite directo. Entonces, $\Gal(E/k)=\varprojlim_{F\in I} \Gal(F/k)$ tanto en el caso finito como infinito.
\end{obs}

\begin{defn}
Sea $E/k$ una extensión Galois. Consideremos: $$\mathscr{B}=\{\sigma\tau : \sigma\in\Gal(E/k),\tau\in\Gal(E/F), F\in I\} = \{\sigma\Gal(E/F):\sigma\in\Gal(E/k),F\in I\}$$ Entonces $\mathscr{B}$ es una base para una topología, que llamaremos la \textbf{topología de Krull}.
\end{defn}

\begin{obs}
Veamos que de hecho $\mathscr{B}$ es la base de una topología. En efecto, es obvio que para todo $x\in\Gal(E/k)$ existe $B\in\mathscr{B}$ pues tomo $B=\id\Gal(E/k)\in\mathscr{B}$. Ahora, para ver que si $\eta\in\sigma_1\Gal(E/F_1)\cap\sigma_2\Gal(E/F_2)$, entonces hay algún $B\in \mathscr{B}$ tal que $\eta\in B\subseteq \sigma_1\Gal(E/F_1)\cap\sigma_2\Gal(E/F_2)$, simplemente notemos que $B=\eta\Gal(E/F_1F_2)$ nos sirve. Esto es pues si $\eta\in \sigma_1\Gal(E/F_1)$, entonces $\sigma_1\Gal(E/F_1) = \eta\Gal(E/F_1)$ y lo mismo vale para $\sigma_2\Gal(E/F_2)=\eta\Gal(E/F_2)$. Luego, sólo debemos notar que $$\sigma_1\Gal(E/F_1)\cap\sigma_2\Gal(E/F_2) = \eta (\Gal(E/F_1)\cap\Gal(E/F_2)) = \eta \Gal(E/F_1F_2)$$
\end{obs}

\begin{prop}
Sea $E/k$ Galois y consideremos $\Gal(E/k)$ con la topología de Krull. Entonces, $\rho:\Gal(E/k)\to \varprojlim_{F\in I}\Gal(F/k)$ dado por $\rho(\sigma) = (\left.\sigma\right|_{F})_{F\in I}$ es un homeomorfismo.
\begin{proof}
En el teorema anterior vimos que, a nivel de conjuntos, es una biyección. Veamos que $\rho$ es continua y abierta. Para ver la continuidad de $\rho$, basta ver que $\rho_F:\Gal(E/k)\to \Gal(F/k)$, $\rho_F(\sigma)=\left.\sigma\right|_F$ es continua para todo $F\in I$. Como $\Gal(F/k)$ es finito, tiene topología discreta, y así para ver que $\rho_F$ es continua bastará ver que la preimagen de cada punto es abierta. Es decir, si $\sigma\in\Gal(F/k)$, quiero ver que $\rho_F^{-1}(\sigma)\subseteq\Gal(E/k)$ es abierto. Pero notemos que $\rho_F^{-1}(\sigma) = \{\tau\in\Gal(E/k) : \left.\tau\right|_F = \sigma\} = \displaystyle\bigcup_{\tau\in\Gal(E/k), \left.\tau\right|_F=\sigma} \tau\Gal(E/F)$, que es abierto por ser una unión de abiertos (son abiertos porque son abiertos de la base).

Resta ver que $\rho$ es abierta. Basta verlo para los abiertos de la base $\mathscr{B}$. Notemos que \begin{align*}\rho(\sigma\Gal(E/F)) &= \{(\left.\sigma\tau\right|_L)_{L\in I} : \left.\tau\right|_{L\cap F} = \left.\sigma\right|_{L\cap F}\; \forall L\in I\} \\ &= \{(\left.\tilde{\tau}\right|_{L})_{L\in I} : \left.\sigma^{-1}\tilde{\tau}\right|_{L\cap F} = \id_{L\cap F} \;\forall L\in I\} \\ &= \{(\left.\tilde{\tau}\right|_{F\in I} : \left.\tilde{\tau}\right|_{L\cap F} = \left.\sigma\right|_{L\cap F} \;\forall L\in I\} \\ &= \pi_F^{-1}(\left.\sigma\right|_F)\end{align*} Pero como $\left.\sigma\right|_F \in\Gal(E/F)$ es abierto (pues tiene la topología discreta), la preimagen por la proyección es abierta en el producto y así en el limite por ser subespacio del producto. Y estamos.
\end{proof}
\end{prop}

\begin{cor}
Sea $E/k$ una extensión Galois. Entonces $\Gal(E/k)$ con la topología de Krull resulta un grupo topológico, Hausdorff, compacto y totalmente disconexo.
\begin{proof}
Es obvio pues el límite directo de grupos finitos cumplía eso y es homeomorfo a un límite directo de grupos finitos.
\end{proof}
\end{cor}

\begin{prop}
Sea $E/k$ Galois y consideremos $\Gal(E/k)$ con la topología de Krull. Si $H\subseteq \Gal(E/k)$ es un subgrupo, entonces $\Gal(E/E^H) = \overline{H}$ la clausura de $H$.
\begin{proof}
Es obvio que $H\subseteq\Gal(E/E^H)$ por definición. Veamos que $\Gal(E/E^H)$ es cerrado, pues esto implicará que $\overline{H}\subseteq\Gal(E/E^H)$. Para eso, veamos que el complemento $\Gal(E/k)-\Gal(E/E^H)$ es abierto. Sea $\sigma\in\Gal(E/k)-\Gal(E/E^H)$. Entonces, existe $x\in E^H$ tal que $\sigma(x)\neq x$. Sabemos que existe $F/k$ Galois y finita tal que $x\in F$. Luego, considero $\sigma\Gal(E/F)\in\mathscr{B}$. Veamos que $\sigma\Gal(E/F)\subseteq \Gal(E/k)-\Gal(E/E^H)$. Pero esto es obvio, pues si $\sigma\tau\in\sigma\Gal(E/F)$, entonces $\sigma\tau(x)=\sigma(x) \neq x$ y así $\sigma\tau\notin \Gal(E/E^H)$. 

Veamos la otra contención. Es decir, $\Gal(E/E^H)\subseteq \overline{H}$. Sea $\sigma\in\Gal(E/E^H)$. Debo ver que $\sigma\in\overline{H}$. Es decir, para cualquier $\tau\Gal(E/F)\in\mathscr{B}$ tal que $\sigma\in\tau\Gal(E/F)$ se tiene que $\tau\Gal(E/F)\cap H\neq \emptyset$. En efecto, si $\sigma\in\tau\Gal(E/F)$, entonces $\tau\Gal(E/F)=\sigma\Gal(E/F)$. Ahora bien, si $\rho_F:\Gal(E/k)\to\Gal(F/k)$, dada por la restricción $\rho_F(\eta)=\left.\eta\right|_F$, tenemos que $H_0=\rho_F(H)\subseteq \Gal(F/k)$. Pero esta extensión $F/k$ es Galois y finita, y así podemos aplicar el Teorema Fundamental de la Teoría de Galois. Es decir, $\Gal(F/F^{H_0})=H_0$. Entonces, como $\sigma\in\Gal(E/E^H)$ y me estoy restringiendo, tengo que $\sigma\in\Gal(F/F^{H_0})=H_0$. Esto es $\rho_F(\sigma)\in\rho_F(H)=H_0$, y así $\left.\sigma\right|_F = \left.\mu\right|_F$ para algún $\mu\in H$. Esto implica que $\left.\sigma^{-1}\mu\right|_F = \id_F$ y así $\sigma^{-1}\mu\in\Gal(E/F)$. Esto es, $\mu = \sigma(\sigma^{-1}\mu)\in\sigma\Gal(E/F)$ y como $\mu \in H$, la intersección contiene a $\mu$ y así es no vacía. Como queríamos probar.
\end{proof}
\end{prop}

\begin{teo}[Teorema de Fundalmental de la Teoría de Galois III]
Sea $E/k$ Galois y consideremos $\Gal(E/k)$ provisto de la topología de Krull. Entonces, existe una correspondencia biunívoca $$\varphi:\{\text{Cuerpos intermedios}\}\longleftrightarrow\{\text{Subgrupos cerrados de } \Gal(E/F)\}$$ dada por $\varphi(F)=\Gal(E/F)$ y $\varphi^{-1}(H)=E^H$. 

Además, si $F$ es un cuerpo intermedio, son equivalentes:
\begin{enumerate}
\item $[F:k]<+\infty$.
\item $(\Gal(E/k):\Gal(E/F))<+\infty$.
\item $\Gal(E/F)$ es abierto en $\Gal(E/k)$.
\end{enumerate}
Finalmente, si $H\subseteq \Gal(E/k)$ es un subgrupo cerrado, entonces $H\triangleleft G$ si y sólo si $F/k$ es Galois.
\end{teo}

\section{Cuerpos Finitos}

\begin{defn}
Un cuerpo finito es un cuerpo cuyo cardinal es finito.
\end{defn}

\begin{obs}
Sea $k$ un cuerpo finito. Entonces, su cuerpo primo debe ser $\mathbb{F}_p$, pues si fuera $\QQ$ entonces sería infinito. Además, como $k$ es finito, la extensión $k/\FF_p$ resulta finita y si $[k:\FF_p]=n$, tendremos que $k$ es un $\FF_p$-espacio vectorial con dimensión $n$ y así $p^n$ elementos.

Sea $k$ un cuerpo con $q=p^n$ elementos. Entonces, $k^\times$ es un grupo multiplicativo de $q-1$ elementos. Esto implica que para todo $\alpha\in k^\times$ tenemos que $\alpha^{q-1}=1$. Luego, $\alpha^q = \alpha$ para todo $\alpha\in k$, y así $\alpha$ es raíz de $x^q -x$. Pero notemos que $x^q - x$ tiene como máximo $q$ raíces en $\overline{\FF}_p$, así que ya debo tener todas las raíces. Es decir, $x^q - x = \displaystyle\prod_{\alpha\in k}(x - \alpha)$, que implica que $k$ es el cuerpo de descomposición de $x^q - x$ sobre $\FF_p$. Entonces, $k$ es único a menos de isomorfismo (si existe) por ser cuerpo de descomposición.

Ahora, veamos que dado $p$ primo y $n\in\NN$, existe un cuerpo de $q=p^n$ elementos. Sea $k=\{\text{raíces de }x^q - x \text{ en }\overline{\FF_p}\}$. Afirmo que $k$ es un cuerpo. En efecto, simplemente es notar que $0,1\in k$ y que si $\alpha,\beta\in k$, la suma está (es aplicar el Frobenius) y el producto y los inversos. Además, como la derivada de $x^q - x$ es $-1\neq 0$, tenemos que $x^q - x$ es coprimo con su derivada y así separable. Entonces, tiene todas sus raíces distintas, lo que implica que hay $q$ elementos en $k$. Es decir, hemos probado el siguiente teorema:
\end{obs}

\begin{teo}
Dado $p$ primo y $n\in\NN$, entonces existe un cuerpo de $q=p^n$ elementos y este cuerpo es el cuerpo de descomposición de $f=x^q - x\in\FF_p[x]$. Por lo tanto, es único a menos de isomorfismo y se lo denota $\FF_q$.
\end{teo}

\begin{prop}
Sea $\FF_q$ el cuerpo finito con $q$ elementos y $n\in\NN$. Entonces, existe una única extensión $E/\FF_q$ de grado $n$ de $\FF_q$. Más aún, $E=\FF_{q^n}$.
\begin{proof}
Sabemos que $q=p^r$ para algún $r\in\NN$. Entonces, $q^n = p^{rn}$. Luego, $E=\FF_{p^{rn}}$ es el único cuerpo de $q^n$ elementos y es claro que $\FF_q\subseteq \FF_{q^n}$ (pues si $x^q = x$ entonces $x^{q^n} = x$). Y listo.
\end{proof}
\end{prop}

\begin{teo}
Sea $q=p^n$. Entonces, $\FF_q/\FF_p$ es Galois y $\Gal(\FF_q/\FF_p)\simeq \ZZ_n$.
\begin{proof}
Notemos que $\FF_q/\FF_p$ es Galois pues es el cuerpo de descomposición de $x^q - x$, que es separable. Ahora, consideremos el Frobenius $\sigma:\FF_q\to\FF_q$, $\sigma(x)=x^p$. Esto está bien definido por la normalidad de $\FF_q/\FF_p$ (o bien porque si $x^q = x$ entonces $(x^p)^q = (x^q)^p = x^p$ y así si $x\in\FF_q$, $x^p$ también está). Como $\FF_q$ es finito y $\sigma$ es inyectivo, entonces $\sigma$ resulta un isomorfismo. Pero además $\left.\sigma\right|_{\FF_p}=\id$. Luego, $\sigma\in\Gal(\FF_q/\FF_p)$. Es claro también que $\sigma^n(x) = x^{p^n} = x$ y así $\sigma^n = \id$. Por otra parte, si $d<n$, existe $x\in\FF_q$ tal que $\sigma^d(x) = x^{p^d} \neq x$ (si no, tendría $p^n$ elementos en $\FF_{p^d}$).  Entonces, $\langle \sigma\rangle \simeq \ZZ_n$. Como $[\FF_q:\FF_p]=n = |\Gal(\FF_q/\FF_p)|$, por ser Galois la extensión, tenemos que $\langle\sigma \rangle = \Gal(\FF_q/\FF_p)$. Y estamos.
\end{proof}
\end{teo}

\begin{lem}
Sea $G$ un grupo cíclico, $G\simeq\langle \sigma\rangle$ de orden $n$. Entonces, para todo $m$ tal que $m\mid n$ existe un único subgrupo $H_m\subseteq G$ de orden $\frac{n}{m}$ y $H_m = \langle \sigma^m\rangle$. En particular, $H_m$ es cíclico.
\begin{proof}
Es claro que $H_m = \langle \sigma^m \rangle$ son cíclicos de orden $\frac{n}{m}$. Ahora, dado $H\subseteq G$, veamos que $H=H_m$ para algún $m\mid n$. Sea $m=\min\{r\in\NN : \sigma^r \in H\}$. Veamos que $H=H_m = \langle \sigma^m\rangle$. En efecto, $\langle\sigma^m\rangle\subseteq H$ pues $\sigma^m\in H$ y si $\tau\in H$, entonces $\tau = \sigma^s$ para algún $s\in\NN$, así que escribiéndo $s=mq+r$, tenemos que $\tau=(\sigma^m)^q\sigma^r$. Esto implica que $\sigma^r \in H$, y así por la minimalidad de $m$ y como $r<m$ tenemos que $r=0$ y así $\tau = \sigma^{mq}\in\langle\sigma^m\rangle$. Con una cuenta similar se ve que $m\mid n$.
\end{proof}
\end{lem}

\begin{cor}
Sea $E=\FF_q$ y $q=p^n$. Entonces, los subcuerpos de $\FF_q$ son de la forma $\FF_{p^m}$ con $m\mid n$.
\begin{proof}
Por el Teorema Fundamental de la Teoría de Galois, los subcuerpos de $\FF_{p^n}/\FF_p$ están en correspondencia con los subgrupos de $\langle\sigma\rangle\simeq\ZZ_n$, y estos a su vez son $\langle\sigma^m\rangle\simeq\ZZ_m$ y así $\FF_q^{\langle \sigma^m\rangle} = \FF_{p^m}$. Y listo.
\end{proof}
\end{cor}

\begin{obs}
Como consecuencia del corolario anterior, se ve que $\FF_{p^m}\subseteq \FF_{p^n}$ si y sólo si $m\mid n$.
\end{obs}

\begin{obs}
Sea $f\in\FF_p[x]$ un polinomio mónico e irreducible de grado $d$. Sea $\alpha\in\overline{\FF_p}$ una raíz de $f$. Entonces, $[\FF_p(\alpha):\FF_p] = \gr f = d$. Como existe una única extensión de grado $d$, tenemos que $\FF_p(\alpha) = \FF_{p^d} = \{\text{raíces de }x^{p^d}-x\}$. Esto implica que $f\mid x^{p^d}-x$, pues $x^{p^d}-x$ anula a $\alpha$ y $f$ es su minimal. Ahora bien, como $x^{p^d}-x$ se factoriza linealmente en $\FF_p(\alpha)$, entonces $f$ también. Entonces, $\FF_p(\alpha)/\FF_p$ es el cuerpo de descomposición de $f$. Es decir, $\FF_{p^d}$ es el cuerpo de descomposición de cualquier polinomio irreducible de grado $d$ de $\FF_p[x]$. Sea $n\in\NN$ con $d\mid n$ y $f\in\FF_p[x]$ irreducible mónico de grado $d$. Entonces, $f\mid x^{p^d}-x \mid x^{p^n}-x$ por ser $\FF_{p^d}\subseteq \FF_{p^n}$. Entonces, todos los irreducibles de grado $d$ son factores de $x^{p^n}-x$ y como $x^{p^n}-x$ es separable, son factores simples. Luego, esto es $x^{p^n} - x = \displaystyle\prod_{d\mid n}\displaystyle\prod_{\stackrel{f\text{ irreducible}}{\text{mónico de grado }d}}f$. Además, si $\psi(d)$ es la cantidad de polinomios mónicos e irreducibles de grado $d$ sobre $\FF_p[x]$, tenemos que (contando los grados en la fórmula anterior), $p^n = \displaystyle\sum_{d\mid n}d\psi(d)$. Usando la Fórmula de Inversión de Möbius, conseguimos la fórmula $\psi(n) = \dfrac{1}{n}\displaystyle\sum_{d\mid n}\mu(d)p^{n/d}$. Si queremos los irreducibles (no sólo mónicos) simplemente multiplicamos esa fórmula por $p-1$.
\end{obs}

\section{Extensiones Ciclotómicas}

\begin{defn}
Sea $n\in\NN$. Una raíz $n$-ésima de la unidad es un elemento $\xi\in k$ tal que $\xi^n =1$. Por convención, si $\car k = p > 0$ pediremos que $p\nmid n$ (esto es porque si $p\mid n$ no tendremos raíces \textit{primitivas}).

Considero el polinomio $f=x^n - 1 \in k[x]$, y como $f'=nx^{n-1}\neq 0$ y no tiene raíces en común con $f$, entonces $f$ debe ser separable. En $\overline{k}$, $f$ tiene $n$ raíces distintas. Las raíces de $f$ en $\overline{k}$ forman un subgrupo finito de $\overline{k}^\times$, que denotaremos $U_n$. Luego, $U_n$ es cíclico y llamamos una \textbf{raíz primitiva $n$-ésima de $1$}.
\end{defn}

\begin{teo}
Sea $\xi_n$ una raíz primitiva $n$-ésima de la unidad. Entonces $k(\xi_n)/k$ es Galois con grupo de Galois $\Gal(k(\xi_n)/k)\subseteq\ZZ_n^\times$. Por lo tanto, la extensión es abeliana y $|\Gal(k(\xi_n)/k)|\mid \varphi(n)$.
\begin{proof}
Primero veamos que $k(\xi_n)/k$ es normal. En efecto, sea $\sigma:k(\xi_n)/k\to\overline{k}/k$ un morfismo de extensiones. Como tenemos que $\sigma(\xi_n)^n = \sigma(\xi_n^n) = \sigma(1)=1$, debe ser que $\sigma(\xi_n)$ es una raíz $n$-ésima de la unidad. Como $\xi_n$ genera a las raíces $n$-ésimas de la unidad, tenemos que $\sigma(\xi_n)=\xi_n^i$ para algún $i$. Entonces, $\sigma$ se correstringe bien y así la extensión es normal. Por otra parte, $k(\xi_n)/k$ resulta separable pues $\xi_n$ es raíz de $x^n - 1$ que es separable. Entonces la extensión es Galois como queríamos.

Además, si $\sigma\in\Gal(k(\xi_n)/k)$, en realidad $\sigma(\xi_n)$ debe ser una raíz $n$-ésima primitiva, pues si $\sigma(\xi_n)^d =1$ para $d<n$, tendremos que $\sigma(\xi_n^d) = 1$ y así como $\sigma$ es monomorfismo, $\xi_n^d = 1$. Absurdo por ser $d<n$. Entonces, si $\sigma(\xi_n)=\xi_n^i$ debemos tener que $i$ es coprimo con $n$, o equivalentemente $i\in\ZZ_n^\times$. Consideremos la función $\gamma:\Gal(k(\xi_n)/k)\to \ZZ_n^\times$ dada por $\gamma(\sigma) = i$ donde $\sigma(\xi_n)=\xi_n^i$. Esta función es de hecho un morfismo de grupos, pues si tomo $\sigma,\tau\in\Gal(k(\xi_n)/k)$ tenemos que $(\sigma\tau)(\xi_n) = \xi_n^{\gamma(\sigma\tau)}$ y a su vez también tenemos que $(\sigma\tau)(\xi_n) = \sigma(\xi_n^{\gamma(\tau)}) = \xi_n^{\gamma(\sigma)\gamma(\tau)}$ de donde $\gamma(\sigma\tau)=\gamma(\sigma)\gamma(\tau)$. Pero además $\gamma$ claramente es monomorfismo. El teorema sigue.
\end{proof}
\end{teo}

\begin{teo}
Si consideramos $\QQ(\xi_n)/\QQ$, entonces $[\QQ(\xi_n):\QQ]=\varphi(n)$ y por lo tanto, $\Gal(\QQ(\xi_n)/\QQ)\simeq\ZZ_n^\times$.
\begin{proof}
Sea $f=m(\xi_n,\QQ)$. Sabemos que $f\mid x^n - 1$ y que $\gr f \leq \varphi(n)$ pues, por el teorema anterior, $|\Gal(\QQ(\xi_n)/\QQ)|\mid \varphi(n)$ y $\gr f = [\QQ(\xi_n):\QQ] = |\Gal(\QQ(\xi_n)/\QQ)|$. Veamos que toda raíz $n$-ésima primitiva de $1$ en $\overline{Q}$ es raíz de $f$. Luego, tendremos al menos $\varphi(n)$ raíces y así $\gr f \geq \varphi(n)$ y listo.

Para probar esto, basta ver que para cada primo con $(p:n)=1$ tenemos que $\xi_n^p$ es raíz de $f$. Sea $p$ primo coprimo con $n$ y supongamos que $\xi_n^p$ no es raíz de $f$. Como tenemos que $x^n -1 = f(x)h(x)$. Como $x^n - 1$ es mónico y $f$ también, entonces $h$ resulta mónico y por Lema de Gauss, resulta que $f,h\in\ZZ[x]$. Como $\xi_n^p$ no es raíz de $f$, debe ser raíz de $h$ (pues es raíz de $x^n - 1$). Luego, $\xi_n$ resulta raíz de $h(x^p)$. Escribo entonces $h(x^p) = f(x)g(x)$, y resulta $g$ mónico. Nuevamente por Lema de Gauss, tenemos que $g\in\ZZ[x]$. Pero notemos que por Fermat, $h(x^p)\equiv (h(x))^p \pmod{p}$. Consideremos la proyección canónica $\ZZ[x]\to\ZZ_p[x]$. Esto implica que $\overline{f}\overline{g} = \overline{h}^p$ y así $\overline{f}$ y $\overline{h}$ no son coprimos. Como $x^n -1 = f(x)h(x)$, resulta que $x^n - \overline{1} = \overline{f}\overline{h}$. Pero entonces $x^n - \overline{1}$ no puede ser separable, pues $\overline{f}$ y $\overline{h}$ no son coprimos. Esto es absurdo, y el teorema sigue.
\end{proof}
\end{teo}

\end{document}


